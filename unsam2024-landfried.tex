\newif\ifen
\newif\ifes
\newif\iffr
\newcommand{\fr}[1]{\iffr#1 \fi}
\newcommand{\En}[1]{\ifen#1\fi}
\newcommand{\Es}[1]{\ifes#1\fi}
\estrue
\documentclass[shownotes,aspectratio=169]{beamer}

\usepackage{siunitx}
\input{diapo_encabezado.tex}
\input{tikzlibrarybayesnet.code.tex}
 \mode<presentation>
 {
 %   \usetheme{Madrid}      % or try Darmstadt, Madrid, Warsaw, ...
 %   \usecolortheme{default} % or try albatross, beaver, crane, ...
 %   \usefonttheme{serif}  % or try serif, structurebold, ...
  \usetheme{Antibes}
  \setbeamertemplate{navigation symbols}{}
 }
\estrue
\usepackage{todonotes}
\setbeameroption{show notes}
%
\newcommand{\gray}{\color{black!55}}
\usepackage{ulem} % sout
\usepackage{mdframed}
\usepackage{comment}
\usepackage{listings}
\lstset{
  aboveskip=3mm,
  belowskip=3mm,
  showstringspaces=true,
  columns=flexible,
  basicstyle={\ttfamily},
  breaklines=true,
  breakatwhitespace=true,
  tabsize=4,
  showlines=true
}


\begin{document}

\color{black!85}
\large


\begin{frame}[plain]
\begin{textblock}{160}(0,0)
\includegraphics[width=1\textwidth]{static/peligro_predador.jpeg}
\end{textblock}
\begin{textblock}{160}(3,3)
\LARGE Apuestas de vida
\end{textblock}

\begin{textblock}{160}(100,86)
\footnotesize \textcolor{white}{Gustavo Landfried. UNSAM - 2024.07.16}
\end{textblock}
\end{frame}



% \begin{frame}[plain,noframenumbering]
%
% % \begin{textblock}{160}(0,0)
% % \includegraphics[width=1.18\textwidth]{../../aux/static/fotosintesis}
% % \end{textblock}
% \begin{textblock}{160}(0,-15)
% \includegraphics[width=1\textwidth]{static/tsimane}
% \end{textblock}
%
%
% % VERSION 2
% % \begin{textblock}{160}(6,36)
% % \LARGE \rotatebox[origin=tr]{18}{\textcolor{black!95}{\fontsize{22}{0}\selectfont \textbf{La función}}}
% % \end{textblock}
% % \begin{textblock}{160}(41,32)
% % \LARGE \rotatebox[origin=tr]{23}{\textcolor{black!95}{\fontsize{22}{0}\selectfont \textbf{de}}}
% % \end{textblock}
% \begin{textblock}{160}(21.5,33)
% \LARGE \rotatebox[origin=tr]{28}{\textcolor{black!95}{\fontsize{22}{0}\selectfont \textbf{Inteligencia natrual}}}
% \end{textblock}
% % \begin{textblock}{160}(68,5.3)
% % \LARGE \rotatebox[origin=tr]{26}{\textcolor{black!95}{\fontsize{22}{0}\selectfont \textbf{epistémica}}}
% % \end{textblock}
% % \begin{textblock}{160}(104,5.5)
% % \LARGE \rotatebox[origin=tr]{8}{\textcolor{black!95}{\fontsize{22}{0}\selectfont \textbf{-}}}
% % \end{textblock}
% % \begin{textblock}{160}(110,3)
% % \LARGE \rotatebox[origin=tr]{-14}{\textcolor{black!95}{\fontsize{22}{0}\selectfont \textbf{evolutiva}}}
% % \end{textblock}
%
% \begin{textblock}{160}(2,82)
% \footnotesize \textcolor{white}{Gustavo Landfried \\ UNSAM - 2024.06.18}
% \end{textblock}
%
%
% %
% % \begin{textblock}{55}[0,0](119,22)
% % \begin{turn}{-57}
% % \parbox{7cm}{\sloppy\setlength\parfillskip{0pt}
% % \textcolor{black!0}{\ \ \ \ \ Unidad \unidad} \\
% % \small\textcolor{black!5}{\hspace{-0.15cm} Apuestas óptimas.} \\
% % \small\textcolor{black!5}{\hspace{-0.85cm} Ventajas a favor de la:} \\
% % \small\textcolor{black!5}{\hspace{-1.45cm} Diversificación (propiedad epistémica)}\\
% % \small\textcolor{black!5}{\hspace{-1.7cm} Cooperación (propiedad evolutiva)}\\
% % \small\textcolor{black!5}{ \hspace{-1.75cm}Especialización (propiedad de especiación)} \\
% % \small\textcolor{black!5}{\hspace{-2cm} Heterogeniedad (propiedad ecológica).\\ }}
% % \end{turn}
% % \end{textblock}
% %
%
% \end{frame}


\begin{frame}[plain]
\begin{textblock}{160}(00,04)
\centering
\LARGE Ciencia
\end{textblock}
\vspace{1cm} \Large

\centering

 La ciencia tiene pretensión de \textbf{verdad}

\vspace{0.8cm}

\pause

 \large Ciencias matemáticas \\
 \large Sistemas axiomáticos \textbf{cerrados} sin incertidumbre\\

 \vspace{0.3cm}

  \pause

\large Ciencias con datos  \\
\large Sistemas naturales \textbf{abiertos} con incertidumbre

\pause
\vspace{1cm}

\Large

¿Qué es una verdad en \\ contextos de incertidumbre?
%
% \pause
% \vspace{0.2cm}
%
%
% Sí. Podemos evitar mentir.

\end{frame}
%
%
% \begin{frame}[plain]
% \begin{textblock}{160}(00,04)
% \centering
% \LARGE ¿Todo vale lo mismo?\\
% \end{textblock}
% \vspace{1cm} \large
%
%
% \only<2->{
% \begin{textblock}{50}(3,26) \centering
% \includegraphics[width=1\textwidth, page={6}]{static/sidewalk_bubblegum_1997_1}
% \end{textblock}}
%  \only<3->{
% \begin{textblock}{50}(55,26) \centering
% \includegraphics[width=1\textwidth, page={6}]{static/sidewalk_bubblegum_1997_2}
% \end{textblock}}
% % \only<4>{
% % \begin{textblock}{50}(107,20) \centering
% % \includegraphics[width=1\textwidth, page={6}]{static/sidewalk_bubblegum_1997_3}
% % \end{textblock}}
% \only<4->{
% \begin{textblock}{50}(107,26) \centering
% \includegraphics[width=1\textwidth, page={6}]{static/sidewalk_bubblegum_1997_4}
% \end{textblock}}
%
% \end{frame}
%
%
% \begin{frame}[plain]
% \begin{textblock}{160}(0,4) \centering
% \LARGE No mentir\\
% \end{textblock}
% \vspace{1cm}
%
% \centering
%
% \Large
%
% Conocemos el significa de no mentir. \pause
%
% \vspace{1cm}
%
% \large
%
% \centering
%
% $\bullet$ No asegurar más de lo que sabemos \pause
%
% $\bullet$ Sin ocultar lo que sí sabemos
%
% \pause \centering \vspace{1cm}
%
% \Large
%
% \textbf{¿Cómo exactamente?}
%
%
% \end{frame}


\begin{frame}[plain]
 \begin{textblock}{160}(0,4)
 \centering \LARGE \only<2>{Certeza absoluta \\}\only<3-5>{Distribución de creencias \\}
 \only<6->{¿Cómo preservamos los acuerdos intersubjetivos?\\}
\end{textblock}
\vspace{1.5cm}
\centering


\only<1-4>{
\begin{textblock}{160}(0,62)
\Large Detrás de una de estas caja hay un regalo. \\[0.1cm]

\large ¿Dónde está el regalo?
\end{textblock}
}

\only<1>{
\begin{textblock}{160}(0,28)
 \scalebox{1.1}{
\tikz{ %
         \node[factor, minimum size=1cm] (p1) {} ;
         \node[factor, minimum size=1cm, xshift=1.5cm] (p2) {} ;
         \node[factor, minimum size=1cm, xshift=3cm] (p3) {} ;


         \node[const, above=of p1, yshift=0.1cm] (np1) {\Large $?$};
         \node[const, above=of p2, yshift=0.1cm] (np2) {\Large $?$};
         \node[const, above=of p3, yshift=0.1cm] (np3) {\Large $?$};
         }
}
\end{textblock}
}

\only<2>{
\begin{textblock}{160}(0,28)
 \scalebox{1.1}{
\tikz{ %
         \node[factor, minimum size=1cm] (p1) {} ;
         \node[factor, minimum size=1cm, xshift=1.5cm] (p2) {} ;
         \node[factor, minimum size=1cm, xshift=3cm] (p3) {} ;


         \node[const, above=of p1, yshift=0.125cm] (np1) {\Large $0$};
         \node[const, above=of p2, yshift=0.125cm] (np2) {\Large $1$};
         \node[const, above=of p3, yshift=0.125cm] (np3) {\Large $0$};
         }
}
\end{textblock}
}

\only<3>{
\begin{textblock}{160}(0,28)
 \scalebox{1.1}{
\tikz{ %
         \node[factor, minimum size=1cm] (p1) {} ;
         \node[factor, minimum size=1cm, xshift=1.5cm] (p2) {} ;
         \node[factor, minimum size=1cm, xshift=3cm] (p3) {} ;


         \node[const, above=of p1, yshift=-0.05cm] (np1) {\Large $1/10$};
         \node[const, above=of p2, yshift=-0.05cm] (np2) {\Large $8/10$};
         \node[const, above=of p3, yshift=-0.05cm] (np3) {\Large $1/10$};
         }
}
\end{textblock}
}


\only<4-5>{
\begin{textblock}{160}(0,28)
 \scalebox{1.1}{
\tikz{ %
         \node[factor, minimum size=1cm] (p1) {} ;
         \node[factor, minimum size=1cm, xshift=1.5cm] (p2) {} ;
         \node[factor, minimum size=1cm, xshift=3cm] (p3) {} ;


         \node[const, above=of p1, yshift=-0.05cm] (np1) {\Large $1/3$};
         \node[const, above=of p2, yshift=-0.05cm] (np2) {\Large $1/3$};
         \node[const, above=of p3, yshift=-0.05cm] (np3) {\Large $1/3$};
         }
}
\end{textblock}
}

\only<5>{
\begin{textblock}{140}(10,64)   \centering \Large
Acuerdo intersubjetivo (no mentir)\\[0.1cm]
\large 1. No asegurar más de lo que sabemos (máximizando incertidumbre) \\
\large 2. Sin ocultar lo que sí sabemos (dada la información disponible)

\end{textblock}
}

\only<6->{
\begin{textblock}{160}(0,28)
 \scalebox{1.1}{
\tikz{ %
         \node[factor, minimum size=1cm] (p1) {} ;
         \node[det, minimum size=1cm, xshift=1.5cm] (p2) {\includegraphics[width=0.03\textwidth]{static/dedo.png}} ;
         \node[factor, minimum size=1cm, xshift=3cm] (p3) {} ;


         \node[const, above=of p1, yshift=-0.05cm] (np1) {\Large $\phantom{/}?\phantom{/}$};
         \node[const, above=of p2, yshift=-0.05cm] (np2) {\Large $\phantom{/}0\phantom{/}$};
         \node[const, above=of p3, yshift=-0.05cm] (np3) {\Large $\phantom{/}?\phantom{/}$};
         }
}
\end{textblock}
}

\end{frame}


\begin{frame}[plain]
\begin{textblock}{160}(0,4)
 \centering \LARGE Modelo causal
 \end{textblock}
 \vspace{-1cm}

 \begin{textblock}{80}(0,24)
 \centering

 \vspace{0.3cm}

 \tikz{
    \only<-2>{\phantom}{\node[latent] (d) {\includegraphics[width=0.10\textwidth]{static/dedo.png}} ;}
    \only<-2>{\phantom}{\node[const,above=of d] (nd) {\Large $s$} ;}
    \node[latent, above=of d, xshift=-1.5cm] (r) {\includegraphics[width=0.12\textwidth]{static/regalo.png}} ;
    \node[const,below=of r] (nr) {\Large $r$} ;
    \only<-1>{\phantom}{\node[latent, fill=black!30, above=of d, xshift=1.5cm] (c) {\includegraphics[width=0.12\textwidth]{static/cerradura.png}} ;}
    \only<-1>{\phantom}{\node[const,below=of c] (nc) {\, \Large $c = 1$} ;}
    \only<-2>{\phantom}{\edge {r,c} {d};}

    \only<-2>{\phantom}{\node[const,below=of d] (modelo) {\large $s \neq r$ \, $s \neq c$} ;}
}
 \end{textblock}


\only<1>{
 \begin{textblock}{160}(80,33)
\scalebox{1.5}{
\tikz{
    \node[factor, minimum size=1cm] (p1) {} ;
    \node[factor, minimum size=1cm, xshift=1.5cm] (p2) {} ;
    \node[factor, minimum size=1cm, xshift=3cm] (p3) {} ;

    \node[const, above=of p1, yshift=.15cm] (fp1) {$1/3$};
    \node[const, above=of p2, yshift=.15cm] (fp2) {$1/3$};
    \node[const, above=of p3, yshift=.15cm] (fp3) {$1/3$};
    \node[const, below=of p2, yshift=-.10cm, xshift=0.3cm] (dedo) {};

    \node[invisible, xshift=4.75cm] (s-dist) {};
    \node[invisible, yshift=-1cm] (s-dist) {};
    \node[invisible, yshift=1.2cm] (s-dist) {};
    }
}
\end{textblock}
}

\only<2>{
 \begin{textblock}{160}(80,33)
\scalebox{1.5}{
\tikz{
    \node[factor, minimum size=1cm] (p1) {\includegraphics[width=0.025\textwidth]{static/cerradura.png}} ;
    \node[factor, minimum size=1cm, xshift=1.5cm] (p2) {} ;
    \node[factor, minimum size=1cm, xshift=3cm] (p3) {} ;

    \node[const, above=of p1, yshift=.15cm] (fp1) {$1/3$};
    \node[const, above=of p2, yshift=.15cm] (fp2) {$1/3$};
    \node[const, above=of p3, yshift=.15cm] (fp3) {$1/3$};
    \node[const, below=of p2, yshift=-.10cm, xshift=0.3cm] (dedo) {};

    \node[invisible, xshift=4.75cm] (s-dist) {};
    \node[invisible, yshift=-1cm] (s-dist) {};
    \node[invisible, yshift=1.2cm] (s-dist) {};
    }
}
\end{textblock}
}


\only<3>{
 \begin{textblock}{160}(80,33)
\scalebox{1.5}{
\tikz{
    \node[factor, minimum size=1cm] (p1) {\includegraphics[width=0.025\textwidth]{static/cerradura.png}} ;
    \node[det, minimum size=1cm, xshift=1.5cm] (p2) {\includegraphics[width=0.03\textwidth]{static/dedo.png}} ;
    \node[factor, minimum size=1cm, xshift=3cm] (p3) {} ;

    \node[const, above=of p1, yshift=.15cm] (fp1) {$\phantom{/}?\phantom{/}$};
    \node[const, above=of p2, yshift=.15cm] (fp2) {$\phantom{/}0\phantom{/}$};
    \node[const, above=of p3, yshift=.15cm] (fp3) {$\phantom{/}?\phantom{/}$};
    \node[const, below=of p2, yshift=-.10cm, xshift=0.3cm] (dedo) {};

    \node[invisible, xshift=4.75cm] (s-dist) {};
    \node[invisible, yshift=-1cm] (s-dist) {};
    \node[invisible, yshift=1.2cm] (s-dist) {};
    }
}
\end{textblock}
}

\end{frame}


\begin{frame}[plain]
\begin{textblock}{160}(0,4)
 \centering \LARGE Modelo causal \\
 \Large \phantom{y datos} Universos paralelos
\only<1-12>{\phantom}{y datos}
 \end{textblock}
 \vspace{-1cm}

 \only<1-3>{
 \begin{textblock}{80}(0,24)
 \centering

 \vspace{0.3cm}

 \tikz{
    {\node[latent] (d) {\includegraphics[width=0.10\textwidth]{static/dedo.png}} ;}
    {\node[const,above=of d] (nd) {\Large $s$} ;}
    \node[latent, above=of d, xshift=-1.5cm] (r) {\includegraphics[width=0.12\textwidth]{static/regalo.png}} ;
    \node[const,below=of r] (nr) {\Large $r$} ;
    {\node[latent, fill=black!30, above=of d, xshift=1.5cm] (c) {\includegraphics[width=0.12\textwidth]{static/cerradura.png}} ;}
    {\node[const,below=of c] (nc) {\, \Large $c = 1$} ;}
    {\edge {r,c} {d};}

    {\node[const,below=of d] (modelo) {\large $s \neq r$ \, $s \neq c$} ;}
}
 \end{textblock}
}

  \only<4-12>{
 \begin{textblock}{80}(0,26)
  \centering
  Creencia$(r,s)$ \\ \vspace{0.3cm}
 \begin{tabular}{c|c|c|c||c} \setlength\tabcolsep{0.4cm}
        & \, $r_1$ \, &  \, $r_2$ \, & \, $r_3$ \, & \\ \hline
  { $s_2$}  & \onslide<5->{$1/6$} & \onslide<7->{$0$} & \onslide<9->{$1/3$} & \onslide<12->{$1/2$} \\ \hline
       {$s_3$} & \onslide<6->{$1/6$} & \onslide<8->{$1/3$} & \onslide<10->{$0$} & \onslide<12->{$1/2$} \\ \hline
              & \onslide<12->{$1/3$} &  \onslide<12->{$1/3$} & \onslide<12->{$1/3$}  & \onslide<12->{$1$} \\
\end{tabular}
\end{textblock}
}

\only<13>{
 \begin{textblock}{80}(0,26)
  \centering
  Creencia$(r,\bm{s_2})$ \\ \vspace{0.3cm}
 \begin{tabular}{c|c|c|c||c} \setlength\tabcolsep{0.4cm}
        & \, $r_1$ \, &  \, $r_2$ \, & \, $r_3$ \, & \\ \hline
        { $s_2$}  & \onslide<6->{$1/6$} & \onslide<8->{$0$} & \onslide<10->{$1/3$} & \onslide<13->{$1/2$} \\ \hline
\end{tabular}
\end{textblock}
}


\only<14->{
 \begin{textblock}{80}(0,26)
  \centering
  Creencia$(r|s_2)$ \\ \vspace{0.3cm}
 \begin{tabular}{c|c|c|c||c} \setlength\tabcolsep{0.4cm}
        & \, $r_1$ \, &  \, $r_2$ \, & \, $r_3$ \, & \phantom{$1/2$}\\ \hline
  { $s_2$}  & \onslide<6->{$1/3$} & \onslide<8->{$0$} & \onslide<10->{$2/3$} & \onslide<13->{$1$} \\ \hline
\end{tabular}
\end{textblock}
}


\only<11-12>{
\begin{textblock}{80}(0,64)
 \centering
% \begin{center}
%  Regla de la suma
%  \end{center}

\normalsize
 Creencia$(s_i) =$ sumar de universos paralelos
 \\

\begin{center} \small
 Predecir con la contribución \\ de todas las hipótesis
\end{center}


\end{textblock}
}

\only<13>{
\begin{textblock}{70}(10,52)
\centering
 \scalebox{1}{
\tikz{
    \node[factor, minimum size=1cm] (p1) {\includegraphics[width=0.07\textwidth]{static/cerradura.png}} ;
    \node[det, minimum size=1cm, xshift=1.5cm] (p2) {\includegraphics[width=0.07\textwidth]{static/dedo.png}} ;
    \node[factor, minimum size=1cm, xshift=3cm] (p3) {} ;

    \node[const, above=of p1, yshift=.15cm] (fp1) {$\phantom{1/6}$};
    \node[const, above=of p2, yshift=.15cm] (fp2) {$\phantom{/0/}$};
    \node[const, above=of p3, yshift=.15cm] (fp3) {$\phantom{1/3}$};
    \node[const, below=of p2, yshift=-.10cm, xshift=0.3cm] (dedo) {};

    \node[invisible, xshift=4.75cm] (s-dist) {};
    \node[invisible, yshift=-1cm] (s-dist) {};
    \node[invisible, yshift=1.2cm] (s-dist) {};
    }
}
\end{textblock}
}

\only<13->{
\begin{textblock}{75}(0,74)
 \centering
% \begin{center}
%  Regla de la suma
%  \end{center}


\begin{center} \small
Preservar la creencia previa que \\ sigue siendo compatible con el dato
\end{center}


\end{textblock}
}


% \only<14>{
% \begin{textblock}{80}(0,44)
%  \centering
%
% %  \begin{equation*}
%
%
% %  \begin{equation*}
% % \overbrace{\text{Creencia}(r_i|s_2)}^{\text{\scriptsize Nueva creencia}} = \frac{\overbrace{\text{Creencia}(r_i,s_2)}^{\hfrac{\text{\scriptsize Creencia en la}}{\text{\scriptsize hipótesis compatible}} }}{\underbrace{\text{Creencia}(s_2)}_{\text{Creencia total compatible}}}
% %  \end{equation*}
%
% \end{textblock}
% }


\only<14>{
\begin{textblock}{70}(10,52)
\centering
 \scalebox{1}{
\tikz{
    \node[factor, minimum size=1cm] (p1) {\includegraphics[width=0.07\textwidth]{static/cerradura.png}} ;
    \node[det, minimum size=1cm, xshift=1.5cm] (p2) {\includegraphics[width=0.07\textwidth]{static/dedo.png}} ;
    \node[factor, minimum size=1cm, xshift=3cm] (p3) {} ;

    \node[const, above=of p1, yshift=.15cm] (fp1) {$1/3$};
    \node[const, above=of p2, yshift=.15cm] (fp2) {$\phantom{/}0\phantom{/}$};
    \node[const, above=of p3, yshift=.15cm] (fp3) {$2/3$};
    \node[const, below=of p2, yshift=-.10cm, xshift=0.3cm] (dedo) {};

    \node[invisible, xshift=4.75cm] (s-dist) {};
    \node[invisible, yshift=-1cm] (s-dist) {};
    \node[invisible, yshift=1.2cm] (s-dist) {};
    }
}
\end{textblock}
}

 \only<2-12>{
\begin{textblock}{80}(70,20) \centering
\scalebox{1.2}{
 \tikz{
 \onslide<2->{
\node[latent, draw=white, yshift=0.8cm] (b0) {$1$};
\node[latent,below=of b0,yshift=0.8cm, xshift=-2cm] (r1) {$r_1$};
{\node[latent,below=of b0,yshift=0.8cm] (r2) {$r_2$}; }
\node[latent,below=of b0,yshift=0.8cm, xshift=2cm] (r3) {$r_3$};
\node[latent, below=of r1, draw=white, yshift=0.7cm] (bc11) {$\frac{1}{3}$};
{\node[latent, below=of r2, draw=white, yshift=0.7cm] (bc12) {$\frac{1}{3}$};}
\node[latent, below=of r3, draw=white, yshift=0.7cm] (bc13) {$\frac{1}{3}$};
}
\onslide<3->{
\node[latent,below=of bc11,yshift=0.7cm, xshift=-0.5cm] (r1d2) {$s_2$};
{\node[latent,below=of bc11,yshift=0.7cm, xshift=0.5cm] (r1d3) {$s_3$};}
{\node[latent,below=of bc12,yshift=0.7cm] (r2d3) {$s_3$};}
\node[latent,below=of bc13,yshift=0.7cm] (r3d2) {$s_2$};
\node[latent,below=of r1d2,yshift=0.7cm,draw=white] (br1d2) {$\only<5>{\bm}{\frac{1}{3}\frac{1}{2}}$};
{\node[latent,below=of r1d3,yshift=0.7cm, draw=white] (br1d3) {$\only<6>{\bm}{\frac{1}{3}\frac{1}{2}}$};}
{\node[latent,below=of r2d3,yshift=0.7cm,draw=white] (br2d3) {$\only<8>{\bm}{\frac{1}{3}}$};}
\node[latent,below=of r3d2,yshift=0.7cm,draw=white] (br3d2) {$\only<9>{\bm}{\frac{1}{3}}$};
}

\node[invisible, left=of r1d2,xshift=-0.1cm] (il) {};
\node[invisible, right=of br3d2,xshift=0.1cm] (il) {};

\onslide<2->{
\edge[-] {b0} {r1,r2,r3};
\edge[-] {r1} {bc11};
\edge[-] {r2} {bc12};
\edge[-] {r3} {bc13};
}
\onslide<3->{
\edge[-] {bc11} {r1d2,r1d3};
\edge[-] {bc12} {r2d3};
\edge[-] {bc13} {r3d2};
\edge[-] {r1d2} {br1d2};
\edge[-] {r1d3} {br1d3};
\edge[-] {r2d3} {br2d3};
\edge[-] {r3d2} {br3d2};
}
}
}
\end{textblock}
}


\only<13->{
\begin{textblock}{80}(70,20) \centering
\scalebox{1.2}{
 \tikz{
\node[latent, draw=white, yshift=0.8cm] (b0) {$1$};
\node[latent,below=of b0,yshift=0.8cm, xshift=-2cm] (r1) {$r_1$};
{\color{gray}\node[latent,draw=gray,below=of b0,yshift=0.8cm] (r2) {$r_2$}; }
\node[latent,below=of b0,yshift=0.8cm, xshift=2cm] (r3) {$r_3$};

% \node[latent, below=of r1, draw=white, yshift=0.8cm] (br1) {$\frac{1}{3}$};
% \node[latent, below=of r2, draw=white, yshift=0.8cm] (br2) {$\frac{1}{3}$};
% \node[latent, below=of r3, draw=white, yshift=0.8cm] (br3) {$\frac{1}{3}$};
% \node[latent,below=of br1,yshift=0.8cm] (c11) {$c_1$};
% \node[latent,below=of br2,yshift=0.8cm] (c12) {$c_1$};
% \node[latent,below=of br3,yshift=0.8cm] (c13) {$c_1$};

\node[latent, below=of r1, draw=white, yshift=0.7cm] (bc11) {$\frac{1}{3}$};
{\color{gray}\node[latent, below=of r2, draw=white, yshift=0.7cm] (bc12) {$\frac{1}{3}$};}
\node[latent, below=of r3, draw=white, yshift=0.7cm] (bc13) {$\frac{1}{3}$};
\node[latent,below=of bc11,yshift=0.7cm, xshift=-0.5cm] (r1d2) {$s_2$};
{\color{gray}\node[latent,draw=gray,below=of bc11,yshift=0.7cm, xshift=0.5cm] (r1d3) {$s_3$};}
{\color{gray}\node[latent, draw=gray,below=of bc12,yshift=0.7cm] (r2d3) {$s_3$};}
\node[latent,below=of bc13,yshift=0.7cm] (r3d2) {$s_2$};

\node[latent,below=of r1d2,yshift=0.7cm,draw=white] (br1d2) {$\frac{1}{3}\frac{1}{2}$};
{\color{gray}\node[latent,below=of r1d3,yshift=0.7cm, draw=white] (br1d3) {$\frac{1}{3}\frac{1}{2}$};}
{\color{gray}\node[latent,below=of r2d3,yshift=0.7cm,draw=white] (br2d3) {$\frac{1}{3}$};}
\node[latent,below=of r3d2,yshift=0.7cm,draw=white] (br3d2) {$\frac{1}{3}$};
\edge[-] {b0} {r1,r3};
\edge[-,draw=gray] {b0} {r2};
% \edge[-] {r1} {br1};
% \edge[-] {r2} {br2};
% \edge[-] {r3} {br3};
% \edge[-] {br1} {c11};
% \edge[-] {br2} {c12};
% \edge[-] {br3} {c13};
\edge[-] {r1} {bc11};
\edge[-,draw=gray] {r2} {bc12};
\edge[-] {r3} {bc13};
\edge[-] {bc11} {r1d2};
\edge[-,draw=gray] {bc11} {r1d3};
\edge[-,draw=gray] {bc12} {r2d3};
\edge[-] {bc13} {r3d2};
\edge[-] {r1d2} {br1d2};
\edge[-,draw=gray] {r1d3} {br1d3};
\edge[-,draw=gray] {r2d3} {br2d3};
\edge[-] {r3d2} {br3d2};
}
}
\end{textblock}
}

\end{frame}
%
%
%
% \begin{frame}[plain]
% \begin{textblock}{160}(0,4)
%  \centering \LARGE Modelos causales \\
% \end{textblock}
% \vspace{1cm}
%
%
% \begin{textblock}{160}(8,22)
% %\onslide<2->{Modelo gráfico} \\ \vspace{0.3cm}
%  \tikz{
%     \node[latent,] (r) {\includegraphics[width=0.06\textwidth]{static/regalo.png}} ;
%     \node[const,above=of r, xshift=-0.2cm, yshift=0.3cm] (titulo) {\Large Modelo gráfico} ;
%     \node[const,left=of r] (nr) {Regalo: \Large $r$\,} ;
%
%     \onslide<2->{
%     \node[latent, below=of r] (d) {\includegraphics[width=0.05\textwidth]{static/dedo.png}} ;
%     \node[const, left=of d] (nd) {Pista: \Large $s$\,} ;
%     \node[const, below=of d, yshift=-0.2cm] (c) {$(s \neq r)$};
%
%     \edge {r} {d};
%     }
% }
% \end{textblock}
%
% \only<1-2>{
% \begin{textblock}{160}(65,33)
% \scalebox{1.5}{
% \tikz{
%     \node[factor, minimum size=1cm] (p1) {} ;
%     \node[factor, minimum size=1cm, xshift=1.5cm] (p2) {} ;
%     \node[factor, minimum size=1cm, xshift=3cm] (p3) {} ;
%
%     \node[const, above=of p1, yshift=.15cm] (fp1) {$1/3$};
%     \node[const, above=of p2, yshift=.15cm] (fp2) {$1/3$};
%     \node[const, above=of p3, yshift=.15cm] (fp3) {$1/3$};
%     \node[const, below=of p2, yshift=-.10cm, xshift=0.3cm] (dedo) {};
%
%     \node[invisible, xshift=4.75cm] (s-dist) {};
%     \node[invisible, yshift=-1cm] (s-dist) {};
%     \node[invisible, yshift=1.2cm] (s-dist) {};
%     }
% }
% \end{textblock}
% }
%
% \only<3>{
% \begin{textblock}{160}(65,33)
% \scalebox{1.5}{
% \tikz{ %
%
%          \node[factor, minimum size=1cm] (p1) {} ;
%          \node[det, minimum size=1cm, xshift=1.5cm] (p2) {\includegraphics[width=0.03\textwidth]{static/dedo.png}} ;
%          \node[factor, minimum size=1cm, xshift=3cm] (p3) {} ;
% %
% %
%          \node[const, above=of p1, yshift=.15cm] (fp1) {$?$};
%          \node[const, above=of p2, yshift=.15cm] (fp2) {$0$};
%          \node[const, above=of p3, yshift=.15cm] (fp3) {$?$};
%          \node[const, below=of p2, yshift=-.10cm, xshift=0.3cm] (dedo) {};
%
% %         \node[const, above=of p2, xshift=.8cm, yshift=.15cm] (fp3) {$66\%$};
% %
%          \node[invisible, xshift=4.75cm] (s-dist) {};
%          \node[invisible, yshift=-1cm] (s-dist) {};
%          \node[invisible, yshift=1.2cm] (s-dist) {};
% %
% %         \plate[color=red] {no} {(p1)} {}; %
% %         \plate {si} {(p2)(p3)} {}; %
%
%         }
% }
% \end{textblock}
% }
%
% \end{frame}
%
% \begin{frame}[plain]
% \begin{textblock}{160}(0,4)
%  \centering \LARGE Modelos causales\\
%  \Large Máxima incertidumbre dado el modelo \\
% \end{textblock}
% \vspace{1cm}
% \vspace{1cm}
%
%
% \only<1-3>{
% \begin{textblock}{160}(8,22)
% %\onslide<2->{Modelo gráfico} \\ \vspace{0.3cm}
%  \tikz{
%     \node[latent,] (r) {\includegraphics[width=0.06\textwidth]{static/regalo.png}} ;
%     \node[const,above=of r, xshift=-0.2cm, yshift=0.3cm] (titulo) {\Large {Modelo gráfico}} ;
%     \node[const,left=of r] (nr) {Regalo: \Large $r$\,} ;
%
%     \node[latent, below=of r] (d) {\includegraphics[width=0.05\textwidth]{static/dedo.png}} ;
%     \node[const, left=of d] (nd) {Pista: \Large $s$\,} ;
%     \node[const, below=of d, yshift=-0.2cm] (c) {$(s \neq r)$};
%
%     \edge {r} {d};
%
% }
% \end{textblock}
% }
%
%
% \only<1->{
% \begin{textblock}{80}(60,20) \centering
% \scalebox{1.1}{
% \tikz{
% \onslide<1->{
% \node[latent, draw=white, yshift=0.6cm] (b0) {$ 1$};
%
% \node[latent,below=of b0,yshift=0.6cm, xshift=-3cm] (r1) {$r_1$};
% \node[latent,below=of b0,yshift=0.6cm] (r2) {$r_2$};
% \node[latent,below=of b0,yshift=0.6cm, xshift=3cm] (r3) {$r_3$};
%
% \node[latent, below=of r1, draw=white, yshift=0.6cm] (br1) {$\frac{1}{3}$};
% \node[latent, below=of r2, draw=white, yshift=0.6cm] (br2) {$\frac{1}{3}$};
% \node[latent, below=of r3, draw=white, yshift=0.6cm] (br3) {$\frac{1}{3}$};
% }
% \onslide<2->{
% \node[latent,below=of br1,yshift=0.6cm, xshift=-0.7cm] (r1d2) {$s_2$};
% \node[latent,below=of br1,yshift=0.6cm, xshift=0.7cm] (r1d3) {$s_3$};
%
% \node[latent,below=of r1d2,yshift=0.6cm,draw=white] (br1d2) {$\frac{1}{3}\frac{1}{2}$};
% \node[latent,below=of r1d3,yshift=0.6cm, draw=white] (br1d3) {$\frac{1}{3}\frac{1}{2}$};
% }
% \onslide<3->{
% \node[latent,below=of br2,yshift=0.6cm, xshift=-0.7cm] (r2d1) {$s_1$};
% \node[latent,below=of br2,yshift=0.6cm, xshift=0.7cm] (r2d3) {$s_3$};
% \node[latent,below=of br3,yshift=0.6cm, xshift=-0.7cm] (r3d1) {$s_1$};
% \node[latent,below=of br3,yshift=0.6cm, xshift=0.7cm] (r3d2) {$s_2$};
%
% \node[latent,below=of r2d1,yshift=0.6cm, draw=white] (br2d1) {$\frac{1}{3}\frac{1}{2}$};
% \node[latent,below=of r2d3,yshift=0.6cm,draw=white] (br2d3) {$\frac{1}{3}\frac{1}{2}$};
% \node[latent,below=of r3d1,yshift=0.6cm, draw=white] (br3d1) {$\frac{1}{3}\frac{1}{2}$};
% \node[latent,below=of r3d2,yshift=0.6cm,draw=white] (br3d2) {$\frac{1}{3}\frac{1}{2}$};
% }
% \onslide<1->{
% \edge[-] {b0} {r1,r2,r3};
% \edge[-] {r1} {br1};
% \edge[-] {r2} {br2};
% \edge[-] {r3} {br3};
% }
% \onslide<2->{
% \edge[-] {br1} {r1d2,r1d3};
% \edge[-] {r1d2} {br1d2};
% \edge[-] {r1d3} {br1d3};
% }
% \onslide<3->{
% \edge[-] {br2} {r2d1, r2d3};
% \edge[-] {br3} {r3d1,r3d2};
% \edge[-] {r2d1} {br2d1};
% \edge[-] {r2d3} {br2d3};
% \edge[-] {r3d1} {br3d1};
% \edge[-] {r3d2} {br3d2};
% }
% }
% }
% \end{textblock}
% }
%
%
% \only<4->{
%  \begin{textblock}{65}(0,24)
%   \centering
%   Creencia$(r,s|\text{Modelo})$ \\ \vspace{0.3cm}
%  \begin{tabular}{c|c|c|c||c} \setlength\tabcolsep{0.4cm}
%         & \, $r_1$ \, &  \, $r_2$ \, & \, $r_3$ \, & \\ \hline
%   $s_1$  & \onslide<5->{$0$} & \onslide<6->{$1/6$} & \onslide<6->{$1/6$} & \\ \hline
%   $s_2$  & \onslide<7->{$1/6$} & \onslide<7->{$0$} & \onslide<7->{$1/6$} &  \\ \hline
%        $s_3$ & \onslide<8->{$1/6$} & \onslide<8->{$1/6$} & \onslide<8->{$0$} &  \\ \hline \hline
%               & & &  & \\
% \end{tabular}
% \end{textblock}
% }
%
%
%
% \only<9->{
%  \begin{textblock}{60}(0,65) \centering
% Creencia conjunta\\
%
% intersubjetiva inicial
%  \end{textblock}
%  }
% \end{frame}
%
%
%
% \begin{frame}[plain]
%  \begin{textblock}{160}(0,4)
%  \centering \Large\hspace{1.4cm}Máxima incertidumbre dado el modelo \only<1-5>{\phantom}{y el dato}
%  \end{textblock}
%
% \vspace{1cm}
%
%  \begin{textblock}{160}(0,15)
%   \centering
%   $\overbrace{\text{Creencia}(r,s|\text{M})}^{\text{\scriptsize De ambas variables}}$ \\ \vspace{0.3cm}
%  \begin{tabular}{c|c|c|c||c} \setlength\tabcolsep{0.4cm}
%      $\phantom{\bm{s_2}}$   & \, $r_1$ \, &  \, $\only<2>{\gray}r_2$ \, & \, $\only<2>{\gray}r_3$ \, &  \phantom{\bm{$1/3$}} \\ \hline
%   $\only<5>{\gray}s_1$ & $\only<5>{\gray}0$ & $\only<2,5>{\gray}1/6$ & $\only<2,5>{\gray}1/6$ & \onslide<4->{$\only<5>{\gray}1/3$} \\ \hline
%   $\only<5>{\bm}{s_2}$ & $1/6$ & $\only<2>{\gray}0$ & $\only<2>{\gray}1/6$ & \onslide<4->{$1/3$} \\ \hline
%   $\only<5>{\gray}s_3$ & $\only<5>{\gray}1/6$ & $\only<2,5>{\gray}1/6$ & $\only<2,5>{\gray}0$ & \onslide<4->{$\only<5>{\gray}1/3$} \\ \hline \hline
%         & \onslide<3->{$\only<5>{\gray}1/3$} & \onslide<3->{$\only<5>{\gray}1/3$} & \onslide<3->{$\only<5>{\gray}1/3$} &  \\
% \end{tabular}
%
% \vspace{0.3cm}
%
% \onslide<2->{
% \begin{align*}
%  \text{Creencia}(r|\text{M}) = \onslide<3->{\sum_s \text{Creencia}(r,s|\text{M})}
% \end{align*}
% }
% \vspace{-0.5cm}
% \onslide<4->{
% \begin{align*}
%  \text{Creencia}(s|\text{M}) = \sum_r \text{Creencia}(r,s|\text{M})
% \end{align*}
% }
% \end{textblock}
%
% \end{frame}
%
%
% \begin{frame}[plain]
%  \begin{textblock}{160}(0,4)
%  \centering \Large\hspace{1.4cm}Máxima incertidumbre dado el modelo y el dato
%  \end{textblock}
%
% \vspace{1cm}
%
% \only<1->{
%  \begin{textblock}{160}(0,15)
%   \centering
%   \only<1-2>{$\overbrace{\text{Creencia}(r,s_2|\text{M})}^{\text{\scriptsize De ambas variables}}$}\only<3->{$\overbrace{\text{Creencia}(r|s_2,\text{M})}^{\text{\scriptsize De ambas variables}}$} \\ \vspace{0.3cm}
%  \begin{tabular}{c|c|c|c||c} \setlength\tabcolsep{0.4cm}
%         $\phantom{\bm{s_2}}$ & \, $r_1$ \, &  \, $r_2$ \, & \, $r_3$ \, &  \phantom{\bm{$1/3$}} \\ \hline
%   &  &  &  & \\ \hline
%   $\bm{s_2}$ & \only<1-2>{$1/6$}\only<3>{$\frac{1}{6}/\frac{1}{3}$}\only<4->{$1/2$} & $0$ & \only<1-2>{$1/6$}\only<3>{$\frac{1}{6}/\frac{1}{3}$}\only<4->{$1/2$} & \only<1>{$1/3$}\only<2>{{$\bm{1/3}$}}\only<3>{$\frac{1}{3}/\frac{1}{3}$}\only<4->{$1$} \\ \hline
%   &  &  & &  \\
% \end{tabular}
% \end{textblock}
% }
%
% \only<1>{
% \begin{textblock}{160}(0,58)
% \begin{equation*}
% \ \phantom{\underbrace{\text{Creencia}(r|s_2,\text{M})}_{\text{Nueva creencia}} =} \hfrac{\overbrace{\text{Creencia}(r, s_2|\text{M})}^{\text{Creencia compatible}}}{\phantom{\underbrace{\text{Creencia}(s_2|\text{M})}_{\text{Creencia total que compatible}}}}
% \end{equation*}
% \end{textblock}
% }
%
% \only<2>{
% \begin{textblock}{160}(0,58)
% \begin{equation*}
% \ \phantom{\underbrace{\text{Creencia}(r|s_2,\text{M})}_{\text{Nueva creencia}} =}\hfrac{\overbrace{\text{Creencia}(r, s_2|\text{M})}^{\text{Creencia compatible}}}{\underbrace{\text{Creencia}(s_2|\text{M})}_{\text{Creencia total que compatible}}}
% \end{equation*}
% \end{textblock}
% }
%
%
% \only<3-4>{
% \begin{textblock}{160}(0,58)
% \begin{equation*}
% \underbrace{\text{Creencia}(r|s_2,\text{M})}_{\text{Nueva creencia}} = \frac{\overbrace{\text{Creencia}(r, s_2|\text{M})}^{\text{Creencia compatible}}}{\underbrace{\text{Creencia}(s_2|\text{M})}_{\text{Creencia total que compatible}}}
% \end{equation*}
% \end{textblock}
% }
%
%
% \only<5->{
% \begin{textblock}{160}(7,57)
% \centering
% \scalebox{1.2}{
% \tikz{ %
%
%          \node[factor, minimum size=1cm] (p1) {} ;
%          \node[det, minimum size=1cm, xshift=1.5cm] (p2) {\includegraphics[width=0.03\textwidth]{static/dedo.png}} ;
%          \node[factor, minimum size=1cm, xshift=3cm] (p3) {} ;
%
%          \node[const, above=of p1, yshift=.15cm] (fp1) {$1/2$};
%          \node[const, above=of p2, yshift=.15cm] (fp2) {$0$};
%          \node[const, above=of p3, yshift=.15cm] (fp3) {$1/2$};
%          \node[const, below=of p2, yshift=-.10cm, xshift=0.3cm] (dedo) {};
%
%          \node[invisible, xshift=4.75cm] (s-dist) {};
%          \node[invisible, yshift=-1cm] (s-dist) {};
%          \node[invisible, yshift=1.2cm] (s-dist) {};
%
%         }
% }
% \end{textblock}
% }
%
% \end{frame}
%


{
\setbeamercolor{background canvas}{bg=orange!10}
\begin{frame}[plain]
\begin{textblock}{160}(0,4)
\centering \LARGE Las reglas de la probabilidad
\end{textblock}
\onslide<1>{
\vspace{1.5cm}



\begin{columns}[t]
\begin{column}{0.5\textwidth}
 \centering


\textbf{Regla del producto}

\begin{equation*}
 P(h|d)  = \frac{P(h,d)}{P(d)}
\end{equation*}

\vspace{0.1cm}

\normalsize
Preservamos la creencia previa que \\
sigue siendo compatible con el dato


 \end{column}
 \begin{column}{0.5\textwidth}

\centering


\textbf{Regla de la suma}

\vspace{0.1cm}

\begin{equation*}
 P(d) = \sum_h P(h,d)
\end{equation*}


\vspace{0.1cm}


 \normalsize
 Predecimos con la contribución \\ de todas las hipótesis.


\end{column}
\end{columns}
}
\end{frame}
}


%
% \begin{frame}[plain]
% \begin{textblock}{160}(0,4)
% \centering \LARGE Teorema de Bayes \\
%  \large La sorpresa como filtro de la creencia previa
% \end{textblock}
%
% \Wider[2cm]{
% \begin{textblock}{160}(0,23)
% \begin{equation*}
% \overbrace{P(\text{Hip\'otesis}_i|\,\text{Datos})}^{\hfrac{\text{\small Creencia}}{\text{\small a posteriori}}} = \frac{\overbrace{P(\text{Datos}\,|\,\text{Hip\'otesis$_i$})}^{\text{\small Predicción}} \overbrace{P(\text{Hip\'otesis}_i)}^{\hfrac{\text{\small Creencia}}{\text{\small a priori}}} }{\underbrace{P(\text{Datos})}_{\hfrac{\text{\small Creencia compatible}}{\text{\small con los datos}} }}
% \end{equation*}
% \end{textblock}
% }
%
% \only<2>{
% \begin{textblock}{160}(0,64)
% \large
% \begin{equation*}
% \underbrace{P(\text{\En{Data}\Es{Datos}} = \{d_1, d_2, \dots \}|\text{Hip\'otesis}_i)}_{\text{\small Predicción de un conjunto de datos}}  =  \underbrace{P(d_1 |\text{Hip\'otesis}_i)}_{\text{\small Predic\En{tion}\Es{ción} 1}} \, \underbrace{P(d_2 | d_1 , \text{Hip\'otesis}_i)}_{\text{\small Predic\En{tion}\Es{ción} 2}} \dots
% \end{equation*}
% \end{textblock}
% }
%
% \end{frame}
%
%
%
% \begin{frame}[plain]
% \begin{textblock}{160}(0,4)
% \centering \LARGE Estimación de habilidad \\
% \Large en la industria del deporte y los videojuegos
% \end{textblock}
%
% \begin{textblock}{160}(0,30)
% \includegraphics[width=1\textwidth,page=1]{static/messi_datos.jpeg}
% \end{textblock}
%
%
% \end{frame}
%
%
%
% \begin{frame}[plain]
% \begin{textblock}{160}(0,4)
% \centering \LARGE Estimación de habilidad \\
% \large TrueSkill
% \end{textblock}
%
%
%  \only<1->{
%  \begin{textblock}{140}(13,24)
%  \normalsize
% \tikz{
%     \node[det, fill=black!10] (r) {$r$} ;
%     \node[const, right=of r] (dr) {\normalsize $r = (d > 0)$};
%
%     \node[latent, above=of r, yshift=-0.45cm] (d) {$d$} ; %
%     \node[const, right=of d] (dd) {\normalsize $d = p_i-p_j$};
%
%     \node[latent, above=of d, xshift=-0.8cm, yshift=-0.45cm] (p1) {$p_a$} ; %
%     \node[latent, above=of d, xshift=0.8cm, yshift=-0.45cm] (p2) {$p_b$} ; %
%
%
%     \node[latent, above=of p1, yshift=-0.35cm] (s1) {$s_a$} ; %
%     \node[latent, above=of p2, yshift=-0.35cm] (s2) {$s_b$} ; %
%     \node[const, right=of s2] (ds2) {$\N(\mu_i,\sigma_i^2)$};
%
%     \node[const, right=of p2] (dp2) {\normalsize $\N(s_i,\beta^2)$};
%
%     \node[const, above=of dr] (r_name) {\small Resultado};
%     \node[const, above=of dd] (d_name) {\small Diferencia};
%     \node[const, above=of dp2] (p_name) {\small Desempeño};
%
%     \node[const, above=of ds2, yshift=0.1cm] (s_name) {\small Habilidad};
%
%     \edge {d} {r};
%     \edge {p1,p2} {d};
%     \edge {s1} {p1};
%     \edge {s2} {p2};
%
% }
% \end{textblock}
% }
%
%
% \only<2->{
% \begin{textblock}{100}(60,22) \centering
% Teorema de Bayes.
%
% \only<2>{\includegraphics[width=0.85\textwidth,page=1]{figuras/posterior_win.pdf}}\only<3>{\includegraphics[width=0.85\textwidth,page=2]{figuras/posterior_win.pdf}}\only<4>{\includegraphics[width=0.85\textwidth,page=3]{figuras/posterior_win.pdf}}\only<5>{\includegraphics[width=0.85\textwidth,page=4]{figuras/posterior_win.pdf}}
% \end{textblock}
% }
%
% \end{frame}
%
%
%
%
%
% \begin{frame}[plain]
% \begin{textblock}{160}(0,4)
% \centering \LARGE TrueSkill Through Time \\
% \Large Modelo de historia completa \\
% \end{textblock}
%
% \begin{textblock}{160}(0,20)
% \begin{figure}[ht!]
%   \centering
%   \scalebox{.95}{
%     \tikz{ %
%       \node[latent] (s10) {$s_{a_0}$} ;
%       %
%       \node[latent,  below=of s10,yshift=-0.7cm] (s11) {$s_{a_1}$} ;
%
%       \node[latent, right=of s11, xshift=3cm] (p11) {$p_{a_1}$} ;
%       %
%       \node[latent, below=of s11,yshift=-0.4cm] (s12) {$s_{a_2}$} ;
%       \node[latent, right=of s12, xshift=3cm] (p12) {$p_{a_2}$} ;
%
%       \node[const, right=of p11,xshift=0.5cm] (r1) {$\bm{>}$} ;
%       \node[const, above=of r1, yshift=0.3cm] (nr1) {\footnotesize \ \  Observed result} ;
%       \node[const, right=of p12,xshift=0.5cm] (r2) {$\bm{<}$} ;
%       \node[const, above=of r2, yshift=0.3cm] (nr2) {\footnotesize \ \ Observed result} ;
%
%       \node[latent, left=of s10, xshift=13.4cm] (s20) {$s_{b_0}$} ;
%       \node[latent, below=of s20,yshift=-0.7cm] (s21) {$s_{b_1}$} ;
%       \node[latent, left=of s21, xshift=-3cm] (p21) {$p_{b_1}$} ;
%
%       \node[latent, below=of s21, yshift=-0.4cm] (s22) {$s_{b_2}$} ;
%       \node[latent, left=of s22, xshift=-3cm] (p22) {$p_{b_2}$} ;
%
%
%       \edge {s10} {s11};
%       \edge {s11} {s12};
%       \edge {s20} {s21};
%       \edge {s21} {s22};
%       \edge {s11} {p11};
%       \edge {s12} {p12};
%       \edge {s21} {p21};
%       \edge {s22} {p22};
%
%       \node[const, right=of s10, yshift=0cm,xshift=-0.15cm ] (wp10) {\includegraphics[page={73},width=.125\linewidth]{figuras/smoothing-por-pasos}} ;
%       \node[const, left=of s20, yshift=0cm,xshift=0.15cm ] (wp20) {\includegraphics[page={73},width=.125\linewidth]{figuras/smoothing-por-pasos}} ;
%
%
%       \node[const, left=of s11, yshift=0.6cm ] (post11) {\only<1-3>{\phantom{\includegraphics[page={1},width=.125\linewidth]{figuras/smoothing-por-pasos}}}\only<4-9>{\includegraphics[page={1},width=.125\linewidth]{figuras/smoothing-por-pasos}}\only<10-13>{\includegraphics[page={13},width=.125\linewidth]{figuras/smoothing-por-pasos}}\only<14>{\includegraphics[page={25},width=.125\linewidth]{figuras/smoothing-por-pasos}}\only<15>{\includegraphics[page={37},width=.125\linewidth]{figuras/smoothing-por-pasos}}\only<16>{\includegraphics[page={49},width=.125\linewidth]{figuras/smoothing-por-pasos}}\only<17>{\includegraphics[page={61},width=.125\linewidth]{figuras/smoothing-por-pasos}} } ;
%       \node[const, right=of s11, yshift=0.6cm, xshift=-0.15cm ] (wp11) {\only<2-7>{\includegraphics[page={2},width=.125\linewidth]{figuras/smoothing-por-pasos}}\only<8-13>{\includegraphics[page={14},width=.125\linewidth]{figuras/smoothing-por-pasos}}\only<14>{\includegraphics[page={26},width=.125\linewidth]{figuras/smoothing-por-pasos}}\only<15>{\includegraphics[page={38},width=.125\linewidth]{figuras/smoothing-por-pasos}}\only<16>{\includegraphics[page={50},width=.125\linewidth]{figuras/smoothing-por-pasos}}\only<17>{\includegraphics[page={62},width=.125\linewidth]{figuras/smoothing-por-pasos}}} ;
%       \node[const, left=of p11, yshift=0.6cm, xshift=0.15cm ] (lh11) {\only<3-8>{\includegraphics[page={3},width=.125\linewidth]{figuras/smoothing-por-pasos}}\only<9-13>{\includegraphics[page={15},width=.125\linewidth]{figuras/smoothing-por-pasos}}\only<14>{\includegraphics[page={27},width=.125\linewidth]{figuras/smoothing-por-pasos}}\only<15>{\includegraphics[page={39},width=.125\linewidth]{figuras/smoothing-por-pasos}}\only<16>{\includegraphics[page={51},width=.125\linewidth]{figuras/smoothing-por-pasos}}\only<17>{\includegraphics[page={63},width=.125\linewidth]{figuras/smoothing-por-pasos}}} ;
%
%       \node[const, left=of s12, yshift=0.6cm ] (post12) {\only<7-12>{\includegraphics[page={4},width=.125\linewidth]{figuras/smoothing-por-pasos}}\only<13>{\includegraphics[page={16},width=.125\linewidth]{figuras/smoothing-por-pasos}}\only<14>{\includegraphics[page={28},width=.125\linewidth]{figuras/smoothing-por-pasos}}\only<15>{\includegraphics[page={40},width=.125\linewidth]{figuras/smoothing-por-pasos}}\only<16>{\includegraphics[page={52},width=.125\linewidth]{figuras/smoothing-por-pasos}}\only<17>{\includegraphics[page={64},width=.125\linewidth]{figuras/smoothing-por-pasos}} } ;
%       \node[const, right=of s12, yshift=0.6cm, xshift=-0.15cm  ] (wp12) {\only<5-10>{\includegraphics[page={5},width=.125\linewidth]{figuras/smoothing-por-pasos}}\only<11-13>{\includegraphics[page={17},width=.125\linewidth]{figuras/smoothing-por-pasos}}\only<14>{\includegraphics[page={29},width=.125\linewidth]{figuras/smoothing-por-pasos}}\only<15>{\includegraphics[page={41},width=.125\linewidth]{figuras/smoothing-por-pasos}}\only<16>{\includegraphics[page={53},width=.125\linewidth]{figuras/smoothing-por-pasos}}\only<17>{\includegraphics[page={65},width=.125\linewidth]{figuras/smoothing-por-pasos}} } ;
%       \node[const, left=of p12, yshift=0.6cm, xshift=0.15cm  ] (lh12) {\only<6-11>{\includegraphics[page={6},width=.125\linewidth]{figuras/smoothing-por-pasos}}\only<12-13>{\includegraphics[page={18},width=.125\linewidth]{figuras/smoothing-por-pasos}}\only<14>{\includegraphics[page={30},width=.125\linewidth]{figuras/smoothing-por-pasos}}\only<15>{\includegraphics[page={42},width=.125\linewidth]{figuras/smoothing-por-pasos}}\only<16>{\includegraphics[page={54},width=.125\linewidth]{figuras/smoothing-por-pasos}}\only<17>{\includegraphics[page={66},width=.125\linewidth]{figuras/smoothing-por-pasos}} } ;
%
%
%       \node[const, right=of s21, yshift=0.6cm ] (post21) {\only<1-3>{\phantom{\includegraphics[page={7},width=.125\linewidth]{figuras/smoothing-por-pasos}}}\only<4-9>{\includegraphics[page={7},width=.125\linewidth]{figuras/smoothing-por-pasos}}\only<10-13>{\includegraphics[page={19},width=.125\linewidth]{figuras/smoothing-por-pasos}}\only<14>{\includegraphics[page={31},width=.125\linewidth]{figuras/smoothing-por-pasos}}\only<15>{\includegraphics[page={43},width=.125\linewidth]{figuras/smoothing-por-pasos}}\only<16>{\includegraphics[page={55},width=.125\linewidth]{figuras/smoothing-por-pasos}}\only<17>{\includegraphics[page={67},width=.125\linewidth]{figuras/smoothing-por-pasos}}}  ;
%       \node[const, left=of s21, yshift=0.6cm, xshift=0.15cm  ] (wp21) {\only<2-7>{\includegraphics[page={8},width=.125\linewidth]{figuras/smoothing-por-pasos}}\only<8-13>{\includegraphics[page={20},width=.125\linewidth]{figuras/smoothing-por-pasos}}\only<14>{\includegraphics[page={32},width=.125\linewidth]{figuras/smoothing-por-pasos}}\only<15>{\includegraphics[page={44},width=.125\linewidth]{figuras/smoothing-por-pasos}}\only<16>{\includegraphics[page={56},width=.125\linewidth]{figuras/smoothing-por-pasos}}\only<17>{\includegraphics[page={68},width=.125\linewidth]{figuras/smoothing-por-pasos}}} ;
%       \node[const, right=of p21, yshift=0.6cm, xshift=-0.15cm  ] (lh21) {\only<3-8>{\includegraphics[page={9},width=.125\linewidth]{figuras/smoothing-por-pasos}}\only<9-13>{\includegraphics[page={21},width=.125\linewidth]{figuras/smoothing-por-pasos}}\only<14>{\includegraphics[page={33},width=.125\linewidth]{figuras/smoothing-por-pasos}}\only<15>{\includegraphics[page={45},width=.125\linewidth]{figuras/smoothing-por-pasos}}\only<16>{\includegraphics[page={57},width=.125\linewidth]{figuras/smoothing-por-pasos}}\only<17>{\includegraphics[page={69},width=.125\linewidth]{figuras/smoothing-por-pasos}} } ;
%
%
%       \node[const, right=of s22, yshift=0.6cm ] (post22) {\only<7-12>{\includegraphics[page={10},width=.125\linewidth]{figuras/smoothing-por-pasos}}\only<13>{\includegraphics[page={22},width=.125\linewidth]{figuras/smoothing-por-pasos}}\only<14>{\includegraphics[page={34},width=.125\linewidth]{figuras/smoothing-por-pasos}}\only<15>{\includegraphics[page={46},width=.125\linewidth]{figuras/smoothing-por-pasos}}\only<16>{\includegraphics[page={58},width=.125\linewidth]{figuras/smoothing-por-pasos}}\only<17>{\includegraphics[page={70},width=.125\linewidth]{figuras/smoothing-por-pasos}}} ;
%       \node[const, left=of s22, yshift=0.6cm, xshift=0.15cm  ] (wp22) {\only<5-10>{\includegraphics[page={11},width=.125\linewidth]{figuras/smoothing-por-pasos}}\only<11-13>{\includegraphics[page={23},width=.125\linewidth]{figuras/smoothing-por-pasos}}\only<14>{\includegraphics[page={35},width=.125\linewidth]{figuras/smoothing-por-pasos}}\only<15>{\includegraphics[page={47},width=.125\linewidth]{figuras/smoothing-por-pasos}}\only<16>{\includegraphics[page={59},width=.125\linewidth]{figuras/smoothing-por-pasos}}\only<17>{\includegraphics[page={71},width=.125\linewidth]{figuras/smoothing-por-pasos}} } ;
%       \node[const, right=of p22, yshift=0.6cm, xshift=-0.15cm  ] (lh22) {\only<6-11>{\includegraphics[page={12},width=.125\linewidth]{figuras/smoothing-por-pasos}}\only<12-13>{\includegraphics[page={24},width=.125\linewidth]{figuras/smoothing-por-pasos}}\only<14>{\includegraphics[page={36},width=.125\linewidth]{figuras/smoothing-por-pasos}}\only<15>{\includegraphics[page={48},width=.125\linewidth]{figuras/smoothing-por-pasos}}\only<16>{\includegraphics[page={60},width=.125\linewidth]{figuras/smoothing-por-pasos}}\only<17>{\includegraphics[page={72},width=.125\linewidth]{figuras/smoothing-por-pasos}}} ;
%
%       \only<4->{\node[const, above=of post11] (npost11) {\scriptsize \only<10>{\phantom}{Posterior}} ;}
%       \only<10>{\node[const, above=of post11] (npost11) {\scriptsize \textbf{Posterior}} ;}
%       \only<2->{\node[const, above=of wp11] (nwp11) {\scriptsize \only<8>{\phantom}{Prior}} ;}
%       \only<8>{\node[const, above=of wp11] (nwp11) {\scriptsize \textbf{Prior}} ;}
%       \only<3->{\node[const, above=of lh11] (nlh11) {\scriptsize \only<9>{\phantom}{Predicción}} ;}
%       \only<9>{\node[const, above=of lh11] (nlh11) {\scriptsize \textbf{Predicción}} ;}
%       \only<4->{\node[const, above=of post21] (npost21) {\scriptsize \only<10>{\phantom}{Posterior}} ;}
%       \only<10>{\node[const, above=of post21] (npost21) {\scriptsize \textbf{Posterior}} ;}
%       \only<2->{\node[const, above=of wp21] (nwp21) {\scriptsize \only<8>{\phantom}{Prior}} ;}
%       \only<8>{\node[const, above=of wp21] (nwp21) {\scriptsize \textbf{Prior}} ;}
%       \only<3->{\node[const, above=of lh21] (nlh21) {\scriptsize \only<9>{\phantom}{Predicción}} ;}
%       \only<9>{\node[const, above=of lh21] (nlh21) {\scriptsize \textbf{Predicción}} ;}
%
%       \only<7->{\node[const, above=of post12] (npost12) {\scriptsize \only<13>{\phantom}{Posterior}} ;}
%       \only<13>{\node[const, above=of post12] (npost12) {\scriptsize \textbf{Posterior}} ;}
%       \only<5->{\node[const, above=of wp12] (nwp12) {\scriptsize \only<11>{\phantom}{Prior}} ;}
%       \only<11>{\node[const, above=of wp12] (nwp12) {\scriptsize \textbf{Prior}} ;}
%       \only<6->{\node[const, above=of lh12] (nlh12) {\scriptsize \only<12>{\phantom}{Predicción}} ;}
%       \only<12>{\node[const, above=of lh12] (nlh12) {\scriptsize \textbf{Predicción}} ;}
%
%       \only<7->{\node[const, above=of post22] (npost22) {\scriptsize \only<13>{\phantom}{Posterior}} ;}
%       \only<13>{\node[const, above=of post22] (npost22) {\scriptsize \textbf{Posterior}} ;}
%       \only<5->{\node[const, above=of wp22] (nwp22) {\scriptsize \only<11>{\phantom}{Prior}} ;}
%       \only<11>{\node[const, above=of wp22] (nwp22) {\scriptsize \textbf{Prior}} ;}
%       \only<6->{\node[const, above=of lh22] (nlh22) {\scriptsize \only<12>{\phantom}{Predicción}} ;}
%       \only<12>{\node[const, above=of lh22] (nlh22) {\scriptsize \textbf{Predicción}} ;}
%
%       \node[const, above=of wp10,yshift=-0.55cm] (nwp10) {\scriptsize Prior} ;
%       \node[const, above=of wp20,yshift=-0.55cm] (nwp20) {\scriptsize Prior} ;
%
%       }
%   }
% \end{figure}
% \end{textblock}
%
% \end{frame}
%
%
% % \begin{frame}[plain]
% % \begin{textblock}{160}(0,4)
% % \centering \LARGE TrueSkill Through Time \\
% % \large Estado del arte en la industria del videojuego.
% % \end{textblock}
% %
% % \begin{textblock}{50}(5,28)
% % \centering
% % \includegraphics[width=0.7\textwidth]{static/julia.png}
% % \end{textblock}
% % \begin{textblock}{50}(55,28)
% % \centering
% % \includegraphics[width=1\textwidth]{static/python.png}
% % \end{textblock}
% % \begin{textblock}{50}(115,28) \raggedright
% % \includegraphics[width=0.55\textwidth]{static/R.png}
% % \end{textblock}
% % \end{frame}
%
%
% \begin{frame}[plain, fragile]
% \begin{textblock}{160}(0,4)
% \centering \LARGE TrueSkill Through Time \\
% \large Estado del arte en estimación de habilidad
% \end{textblock}
%
% \begin{textblock}{155}(0,20) \centering
% \tikz{
%   \node[det, draw=white, minimum size=8cm] (x) {.\hspace{16cm}.};
% }
% \end{textblock}
% \begin{textblock}{155}(0,20) \centering
% \includegraphics[width=0.9\textwidth]{figuras/atp.pdf}
% \end{textblock}
%
%
% \end{frame}
%



\begin{frame}[plain]
\begin{textblock}{160}(0,4)
\centering \LARGE Toma de decisiones \\
\large Las funciones de costo
\end{textblock}


\only<1->{
\begin{textblock}{160}(0,24) \centering

\Large \only<1>{La función de costo epistémica\phantom{evolutiva}}\only<2->{La función de costo evolutiva\phantom{epistémica}}

\large
\only<1>{\begin{equation*}
\underbrace{P(\text{Hipótesis},\text{\En{Data}\Es{Datos}})}_{\hfrac{\text{\footnotesize\En{Initial belief compatible}\Es{Creencia compatible }}}{\text{\footnotesize \En{with the data}\Es{con los datos}}}} = \underbrace{P(\text{Hipótesis})}_{\hfrac{\text{\footnotesize\En{Initial intersubjective}\Es{Acuerdo inter-}}}{\text{\footnotesize\En{agreement}\Es{subjetivo inicial}}}} \ \underbrace{P(\text{dato}_1 |\text{Hipótesis})}_{\text{\footnotesize Predic\En{tion}\Es{ción} 1}} \, \underbrace{P(\text{dato}_2 | \text{dato}_1 , \text{Hipótesis})}_{\text{\footnotesize Predic\En{tion}\Es{ción} 2}} \dots
\end{equation*}
}\only<2->{\begin{equation*}
\underbrace{\text{P}(\text{Variante},\text{\En{Data}\Es{Datos}})}_{\hfrac{\text{\footnotesize\En{Initial belief compatible}\Es{Tamaño actual}}}{\text{\footnotesize \En{with the data}\Es{de la población}}}} = \underbrace{\text{P}(\text{Variante})}_{\hfrac{\text{\footnotesize\En{Initial intersubjective}\Es{Tamaño inicial}}}{\text{\footnotesize\En{agreement}\Es{de la población}}}} \underbrace{\text{ R}(\text{dato}_1|\text{Variante})}_{\text{\footnotesize Reproducción $\geq 1$}} \, \underbrace{\text{ S}(\text{dato}_2|\text{dato}_1,\text{Variante}) }_{\text{\footnotesize $0 \leq$ Supervivencia $\leq 1$  }} \dots
\end{equation*}
}

\end{textblock}
}



\only<3->{
\begin{textblock}{160}(0,60) \centering \Large
Un 0 en la secuencia de tasas de reproducción y

supervivencia produce una extinción irreversible

\vspace{0.6cm}

\only<4>{
\textbf{¿Cuáles son las variantes que más crecen?}
}

\end{textblock}
}

\end{frame}



\begin{frame}[plain]
\begin{textblock}{160}(0,4)
\centering \LARGE Apuestas de vida\\
\large Ejemplo
\end{textblock}

\only<1-2>{
\begin{textblock}{160}(0,22) \centering
Con \textbf{Caras}: Reproducción $= 1.5$ \ $(+50\%)$

Con \textbf{Sello}: Supervivencia $=0.6$ \  $(-40\%)$
\end{textblock}
%
\begin{textblock}{160}(0,34) \centering
\includegraphics[width=0.5\textwidth]{static/plata-potosi.jpg}
\end{textblock}
}

\only<2>{
\begin{textblock}{160}(0,80) \centering
\Large \textbf{¿Nos conviene jugar?}
\end{textblock}
}

\only<3>{
\begin{textblock}{160}(0,20) \centering
\includegraphics[width=0.66\textwidth, page = 7]{figuras/apuestasParalelas.pdf}
\end{textblock}
}
\only<4>{
\begin{textblock}{160}(0,20) \centering
\includegraphics[width=0.66\textwidth, page = 8]{figuras/apuestasParalelas.pdf}
\end{textblock}
}

\end{frame}


\begin{frame}[plain]
\begin{textblock}{160}(0,4)
\centering \LARGE La estrategia de la vida \\
\large Transiciones evolutivas mayores
\end{textblock}


\only<1>{
\begin{textblock}{160}(0,28) \centering
\begin{figure}[ht!]
    \centering
  \scalebox{1.2}{
  \tikz{
      \node[accion] (i1) {} ;
      \node[accion, yshift=0.6cm, xshift=0.4cm] (i2) {} ;
      \node[accion, yshift=0.6cm, xshift=-0.4cm] (i3) {} ;
      \node[const, yshift=0.3cm, xshift=0.4cm] (i) {};

      \node[const, yshift=-0.8cm] (ni) {$\hfrac{\text{Individuos}}{\text{solitarios}}$};

      \node[const, yshift=1.2cm, xshift=1.5cm] (m1) {$\hfrac{\text{Formación}}{\text{de grupos}}$};

      \node[const, right=of i, xshift=2cm] (c) {};
      \node[accion, below=of c, yshift=0.35cm, xshift=0.4cm] (c1) {} ;
      \node[accion, above=of c, yshift=-0.35cm, xshift=0.6cm] (c2) {} ;
      \node[accion, above=of c, yshift=-0.35cm, xshift=0.2cm] (c3) {} ;
      \node[const, right=of c, xshift=0.6cm] (cc) {};

      \node[const, right=of ni, xshift=1.3cm] (nc) {$\hfrac{\text{Grupos}}{\text{cooperativos}}$};

      \node[const, right=of m1, xshift=1.2cm] (m2) {$\hfrac{\text{Transición}}{\text{mayor}}$};

      \node[const, right=of cc, xshift=2cm] (t) {};
      \node[accion, below=of t, yshift=0.35cm, xshift=0.4cm] (t1) {} ;
      \node[accion, above=of t, yshift=-0.35cm, xshift=0.6cm] (t2) {} ;
      \node[accion, above=of t, yshift=-0.35cm, xshift=0.2cm] (t3) {} ;

      \node[const, right=of nc, xshift=1.1cm] (nt) {$\hfrac{\text{Unidad de}}{\text{nivel superior}}$};

      \edge {i} {c};
      \edge {cc} {t};

      \plate {transition} {(t1)(t2)(t3)} {}; %
      }
  }
\end{figure}
\end{textblock}
}


\only<2->{
\begin{textblock}{160}(0,22) \centering \Large

La emergencia de unidades cooperativas de

nivel superior es un fenómeno permanente

\vspace{0.8cm} \large

\only<3->{
Nuestra propia vida depende de al menos 4 niveles:
\begin{figure}[H]
\centering
 \begin{subfigure}[b]{0.25\textwidth} \centering
 \onslide<4->{\includegraphics[width=1\linewidth]{static/cloroplastos.jpg}
  \caption*{\En{Eukaryotic cells}\Es{Células eucariota}}}
  \end{subfigure}
 \begin{subfigure}[b]{0.23\textwidth} \centering
  \onslide<5->{\includegraphics[width=1\linewidth]{static/fotosintesis.jpg}
  \caption*{\En{Organisms}\Es{Organismos}}}
  \end{subfigure}
  \begin{subfigure}[b]{0.235\textwidth} \centering
 \onslide<6->{\includegraphics[width=1\linewidth]{static/hormigas2.jpg}
  \caption*{\En{Societies}\Es{Sociedades}}}
 \end{subfigure}
 \begin{subfigure}[b]{0.235\textwidth} \centering
 \onslide<7->{\includegraphics[width=1\linewidth]{static/tsimane2.jpg}
  \caption*{\En{Ecosystems}\Es{Ecosistemas}}}
 \end{subfigure}
\end{figure}
}
\end{textblock}
}

\end{frame}


\begin{frame}[plain]
\begin{textblock}{160}(0,4)
\centering \LARGE \only<1>{Sin cooperación}\only<2->{Cooperación}
\end{textblock}


\only<1>{
\begin{textblock}{160}(0,24)
\centering
  \begin{tabular}{|l|c|c|c|c|c|}
     \hline
         & {\small $\omega_0$} & {\small \  $\Delta$}  & {\small \ \, $\omega_1$ \ } & {\small \  $\Delta$}  & {\small \,  $\omega_2$ }  \\ \hline \hline
        A no-coop& $1$ & $1.5$ &  $1.5$ & $0.6$ & $\bm{0.9}$ \\ \hline
        B no-coop & $1$ & $0.6$ & $0.6$ & $1.5$ & $\bm{0.9}$ \\ \hline
\end{tabular}
\end{textblock}
}
\only<2>{
\begin{textblock}{160}(0,24)
\centering
  \begin{tabular}{|l|c|c|c|c|c|}
     \hline
         & {\small $\omega_0$} & {\small \  $\Delta$}  & {\small \ \, $\omega_1$ \ } & {\small \  $\Delta$}  & {\small \,  $\omega_2$ }  \\ \hline \hline
        A no-coop& $1$ & $1.5$ &  $1.5$ & $0.6$ & $\bm{0.9}$ \\ \hline
        B no-coop & $1$ & $0.6$ & $0.6$ & $1.5$ & $\bm{0.9}$ \\ \hline\hline
        A coop & $1$ & $1.5$ & $1.05$ & $0.6$ & $\bm{1.1}$ \\ \hline
        B coop & $1$ & $0.6$ & $1.05$ & $1.5$ & $\bm{1.1}$\\ \hline
\end{tabular}
\end{textblock}
}

\only<3>{
\begin{textblock}{100}(30,20) \centering
\includegraphics[page=10,width=1\textwidth]{figuras/apuestasParalelas.pdf}
\end{textblock}
}

\only<3>{
\begin{textblock}{100}(130,22)
Cooperación
\end{textblock}
}
\only<3>{
\begin{textblock}{100}(130,68)
Individual
\end{textblock}
}

\end{frame}




\begin{frame}[plain]
\begin{textblock}{160}(0,4)
\centering \LARGE ¿Cooperar o no cooperar?
\end{textblock}


 \only<1-2>{
\begin{textblock}{160}(0,22) \centering \Large
\textbf{¿Y si dejamos de aportar al fondo común \\
y seguimos recibiendo la cuota común?}
\end{textblock}
}


\only<2>{
\begin{textblock}{160}(0,44) \centering
Tragedia de los comunes
 \begin{equation*}
  \bordermatrix{\hspace{1.1cm} _{\text{\tiny Focal}}{\rotatebox{45}{\text{$\mid$}}}^{\text{\tiny \En{Other}\Es{Otro}}} \hspace{-1.5cm} &  \text{Coopera}  & \text{Deserta} \cr
      \ \ \ \text{Coopera} & \text{beneficio}-\text{costo} & -\text{costo} \cr
      \ \ \ \text{Deserta} & \text{beneficio} & 0 }
\end{equation*}
\end{textblock}
}

\only<3>{
\begin{textblock}{120}(20,14) \centering
\includegraphics[page=1,width=1\textwidth]{figuras/dilema.pdf}
\end{textblock}
}
\only<4>{
\begin{textblock}{120}(20,14) \centering
\includegraphics[page=2,width=1\textwidth]{figuras/dilema.pdf}
\end{textblock}
}
\only<5->{
\begin{textblock}{120}(20,14) \centering
\includegraphics[page=3,width=1\textwidth]{figuras/dilema.pdf}
\end{textblock}
}



\end{frame}



\begin{frame}[plain,noframenumbering]
\centering \vspace{0.5cm}
\includegraphics[width=0.66\textwidth]{static/Metodos2.png}
\end{frame}











\end{document}



