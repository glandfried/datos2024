\newif\ifen
\newif\ifes
\newif\iffr
\newcommand{\fr}[1]{\iffr#1 \fi}
\newcommand{\En}[1]{\ifen#1\fi}
\newcommand{\Es}[1]{\ifes#1\fi}
\estrue
\documentclass[shownotes,aspectratio=169]{beamer}

\usepackage{siunitx}
\input{diapo_encabezado.tex}
\input{tikzlibrarybayesnet.code.tex}
 \mode<presentation>
 {
 %   \usetheme{Madrid}      % or try Darmstadt, Madrid, Warsaw, ...
 %   \usecolortheme{default} % or try albatross, beaver, crane, ...
 %   \usefonttheme{serif}  % or try serif, structurebold, ...
  \usetheme{Antibes}
  \setbeamertemplate{navigation symbols}{}
 }
\estrue
\usepackage{todonotes}
\setbeameroption{show notes}
%
\newcommand{\gray}{\color{black!55}}
\usepackage{ulem} % sout
\usepackage{mdframed}
\usepackage{comment}
\usepackage{listings}
\lstset{
  aboveskip=3mm,
  belowskip=3mm,
  showstringspaces=true,
  columns=flexible,
  basicstyle={\ttfamily},
  breaklines=true,
  breakatwhitespace=true,
  tabsize=4,
  showlines=true
}


\begin{document}

\color{black!85}
\large

\begin{frame}[plain,noframenumbering]


\begin{textblock}{160}(0,0)
\includegraphics[width=1\textwidth]{static/deforestacion}
\end{textblock}

\begin{textblock}{80}(22,10)
\textcolor{black!15}{\fontsize{44}{55}\selectfont Causal}
\end{textblock}

\begin{textblock}{47}(95,70)
\centering \textcolor{black!15}{{\fontsize{52}{65}\selectfont Artificial}}
\end{textblock}

\begin{textblock}{80}(100,28)
\LARGE  \textcolor{black!15}{\rotatebox[origin=tr]{-3}{\scalebox{9}{\scalebox{1}[-1]{$p$}}}}
\end{textblock}


\begin{textblock}{80}(66,43)
\LARGE  \textcolor{black!15}{\scalebox{6}{$=$}}
\end{textblock}

\begin{textblock}{80}(36,29)
\LARGE  \textcolor{black!15}{\scalebox{9}{$p$}}
\end{textblock}

 \vspace{2cm}
\maketitle



\begin{textblock}{160}(01,67)
\normalsize \textcolor{black!5}{Inteligencia Artificial vs Causal}
\end{textblock}

% Lugar
\begin{textblock}{160}(01,73)
\scriptsize \textcolor{black!5}{
Gustavo Landfried. \\
Jueves 12 de septiembre 2024 \\
Laboratorios de Métodos Bayesianos. \\
Facultad de Ciencias Exactas y Naturales \\
Universidad de Buenos Aires, Argentina}
\end{textblock}

\end{frame}



\begin{frame}[plain]
\begin{textblock}{170}(00,0) \centering
\includegraphics[width=1\textwidth]{static/agua.jpg}
\end{textblock}

\begin{textblock}{160}(00,04) \centering
\LARGE Las cualidades de los sistemas naturales \\ \Large
emergen de la interacción entre sus partes
\end{textblock}
\vspace{1cm} \Large



\begin{textblock}{155}(00,26) \centering
\includegraphics[width=0.7\textwidth]{static/atomo_de_oxigeno.png}
\end{textblock}

\end{frame}


\begin{frame}[plain]
\begin{textblock}{160}(0,4)
 \centering \LARGE Cooperación \\
 \Large Células que viven en células
\end{textblock}
\vspace{1.3cm} \centering

\includegraphics[width=1\textwidth]{static/cloroplastos}

\end{frame}

\begin{frame}[plain]
\begin{textblock}{160}(0,4)
 \centering \LARGE Cooperación \\
 \Large Organismos multicelulares
\end{textblock}
\vspace{1.3cm} \centering

\includegraphics[width=1\textwidth]{static/fotosintesis}

\end{frame}

\begin{frame}[plain]
\begin{textblock}{160}(0,4)
 \centering \LARGE Cooperación \\
 \Large Sistemas sociales
\end{textblock}
\vspace{1.3cm} \centering

\includegraphics[width=1\textwidth]{static/hormigas2}

\end{frame}

\begin{frame}[plain]
\begin{textblock}{160}(0,4)
 \centering \LARGE Cooperación \\
 \Large Ecosistema
\end{textblock}
\vspace{1.5cm} \centering

\includegraphics[width=0.75\textwidth]{static/tsimane}

\end{frame}


\begin{frame}[plain]
\begin{textblock}{160}(0,4)
 \centering \LARGE La complejidad de la vida
\end{textblock}
\vspace{1.7cm} \centering




\scalebox{1.5}{
\tikz{
    \node[accion] (i1) {} ;
    \node[accion, yshift=0.6cm, xshift=0.4cm] (i2) {} ;
    \node[accion, yshift=0.6cm, xshift=-0.4cm] (i3) {} ;
    \node[const, yshift=0.3cm, xshift=0.4cm] (i) {};

    \node[const, yshift=-0.8cm] (ni) {$\hfrac{\text{Individuos}}{\text{solitarios}}$};

    \node[const, yshift=1.2cm, xshift=1.5cm] (m1) {$\hfrac{\text{Formación}}{\text{de grupos}}$};

    \node[const, right=of i, xshift=2cm] (c) {};
    \node[accion, below=of c, yshift=0.35cm, xshift=0.4cm] (c1) {} ;
    \node[accion, above=of c, yshift=-0.35cm, xshift=0.6cm] (c2) {} ;
    \node[accion, above=of c, yshift=-0.35cm, xshift=0.2cm] (c3) {} ;
    \node[const, right=of c, xshift=0.6cm] (cc) {};

    \node[const, right=of ni, xshift=1.3cm] (nc) {$\hfrac{\text{Grupos}}{\text{cooperativos}}$};

    \node[const, right=of m1, xshift=1.2cm] (m2) {$\hfrac{\text{Transición}}{\text{mayor}}$};

    \node[const, right=of cc, xshift=2cm] (t) {};
    \node[accion, below=of t, yshift=0.35cm, xshift=0.4cm] (t1) {} ;
    \node[accion, above=of t, yshift=-0.35cm, xshift=0.6cm] (t2) {} ;
    \node[accion, above=of t, yshift=-0.35cm, xshift=0.2cm] (t3) {} ;

    \node[const, right=of nc, xshift=1.1cm] (nt) {$\hfrac{\text{Unidad de}}{\text{nivel superior}}$};

    \edge {i} {c};
    \edge {cc} {t};

    \plate {transition} {(t1)(t2)(t3)} {}; %
    }
}


\end{frame}

\begin{frame}[plain,noframenumbering]
\centering \vspace{0.5cm}
\includegraphics[width=0.66\textwidth]{static/Metodos2.png}
\end{frame}











\end{document}



