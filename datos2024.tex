\newif\ifen
\newif\ifes
\newif\iffr
\newcommand{\fr}[1]{\iffr#1 \fi}
\newcommand{\En}[1]{\ifen#1\fi}
\newcommand{\Es}[1]{\ifes#1\fi}
\estrue
\documentclass[shownotes,aspectratio=169]{beamer}

\usepackage{siunitx}

\usepackage{ragged2e} %\justifying
\usepackage{paracol}
\usepackage[utf8]{inputenc} %Para acentos en UTF8 (Prueba: á é í ó ú Á É Í Ó Ú ñ Ñ)
\usepackage{url}
%\usepackage{mathtools}
\usepackage{graphicx}
\usepackage{caption}
\usepackage{float} % para que los gr\'aficos se queden en su lugar con [H]
\usepackage[fleqn]{mathtools} % \coloneqq, flalign
\usepackage{subcaption}
\usepackage{wrapfig}
\usepackage{soul,color} %\st{Hellow world}
\usepackage[fleqn]{amsmath} %para escribir funci\'on partida
\usepackage{blkarray}
\usepackage{hyperref} % para inlcuir links dentro del texto
\usepackage{tabu} 
\usepackage{comment}
\usepackage{amsfonts} % mathbb{N} -> conjunto de los n\'umeros naturales  
\usepackage{enumerate}
\usepackage{listings}
\usepackage[shortlabels]{enumitem} %  shortlabels option to have compatibility with the enumerate-like scheme for label
\usepackage{framed}
\usepackage{mdframed}
\usepackage{multicol}
\usepackage{transparent} % \transparent{1.0}
\usepackage{bm} 
\usepackage[makeroom]{cancel} % \cancel{} \bcancel{} etc
\usepackage[absolute,overlay]{textpos} %no funciona
\setlength{\TPHorizModule}{1mm} %128mm  mitad: 64 
\setlength{\TPVertModule}{1mm}	%96mm  mitad 48

\newcommand\hfrac[2]{\genfrac{}{}{0pt}{}{#1}{#2}} %\frac{}{} sin la linea del medio


\usepackage{bm} % \bm{\alpha} bold greek symbol
\usepackage[makeroom]{cancel} % \cancel{} \bcancel{} etc

\newcommand{\vm}[1]{\mathbf{#1}}
\newcommand{\N}{\mathcal{N}}
\newcommand{\citel}[1]{\cite{#1}\label{#1}}
\newcommand{\bet}{\rotatebox[origin=tr]{-3}{\scalebox{1}[-1]{$p$}}}


\newtheorem{midef}{Definition}
\newtheorem{miteo}{Theorem}
\newtheorem{mipropo}{Proposition}

\usefonttheme[onlymath]{serif}


\usepackage{tikz} % Para graficar, por ejemplo bayes networks
%\usetikzlibrary{bayesnet} % Para que ande se necesita copiar el archivo  tikzlibrarybayesnet.code.tex en la misma carpeta

%%%%%%%%%%%%%%%%%%%%%%%%%%%%%%%%%5
%
% Incompatibles con textpos
%
%\usepackage{todonotes}
%\usepackage{tikz} % Para graficar, por ejemplo bayes networks
%
%%%%%%%%%%%%%%%%%%%%%%%%%%%%%%%%%%



\usepackage[absolute,overlay]{textpos} %no funciona
\setlength{\TPHorizModule}{1mm} %128mm  mitad: 64 
\setlength{\TPVertModule}{1mm}	%96mm  mitad 48
% 
% 


% Traductor de documentos a través de doble escritura.
\newif\ifen
\newif\ifes
\newcommand{\en}[1]{\ifen#1\fi}
\newcommand{\es}[1]{\ifes#1\fi}

\newcommand{\E}{\en{S}\es{E}}
\newcommand{\A}{\en{E}\es{A}}
\newcommand{\Ee}{\en{s}\es{e}}
\newcommand{\Aa}{\en{e}\es{a}}


\captionsetup[figure]{labelformat=empty}

\hypersetup{
    colorlinks=true,       % false: boxed links; true: colored links
    linkcolor=white,          % color of internal links (change box color with linkbordercolor)
    citecolor=green,        % color of links to bibliography
    filecolor=magenta,      % color of file links
    urlcolor=cyan           % color of external links
}
% 
% http://latexcolor.com/
\definecolor{lightseagreen}{rgb}{0.13, 0.7, 0.6.5}
\definecolor{greenblue}{rgb}{0.1, 0.55, 0.5}
\definecolor{redgreen}{rgb}{0.6, 0.4, 0.}
\definecolor{greenred}{rgb}{0.4, 0.7, 0.}
\definecolor{redblue}{rgb}{0.4, 0., .4}
\definecolor{tangelo}{rgb}{0.98, 0.3, 0.0}
\definecolor{git}{rgb}{0.94, 0.309, 0.2}
% 
\setbeamercolor{structure}{fg=greenblue}


%http://latexcolor.com/
\definecolor{azul}{rgb}{0.36, 0.54, 0.66}
\definecolor{rojo}{rgb}{0.7, 0.2, 0.116}
\definecolor{rojopiso}{rgb}{0.8, 0.25, 0.17}
\definecolor{verdeingles}{rgb}{0.12, 0.5, 0.17}
\definecolor{ubuntu}{rgb}{0.44, 0.16, 0.39}
\definecolor{debian}{rgb}{0.84, 0.04, 0.33}
\definecolor{dkgreen}{rgb}{0,0.6,0}
\definecolor{gray}{rgb}{0.5,0.5,0.5}
\definecolor{mauve}{rgb}{0.58,0,0.82}




\newcommand\Wider[2][3em]{%
\makebox[\linewidth][c]{%
  \begin{minipage}{\dimexpr\textwidth+#1\relax}
  \raggedright#2
  \end{minipage}%
  }%
}

\newenvironment{ejercicio}[1]{
% \setbeamercolor{block title}{bg=tangelo, fg=white}
\begin{exampleblock}{#1}
}{
\end{exampleblock}
}

\newenvironment{resumen}[1]{
\setbeamercolor{block title}{bg=git, fg=white}
\begin{block}{#1}
}{
\end{block}
}

\newenvironment{comando}{
\setbeamercolor{block body}{bg=git, fg=white}
\begin{block}{}
\begin{center}
\LARGE
\begin{texttt}
}{
\end{texttt}
\end{center}
\end{block}
}

%\newcommand{\N}{\mathcal{N}}

% tikzlibrary.code.tex
%
% Copyright 2010-2011 by Laura Dietz
% Copyright 2012 by Jaakko Luttinen
%
% This file may be distributed and/or modified
%
% 1. under the LaTeX Project Public License and/or
% 2. under the GNU General Public License.
%
% See the files LICENSE_LPPL and LICENSE_GPL for more details.

% Load other libraries

%\newcommand{\vast}{\bBigg@{2.5}}
% newcommand{\Vast}{\bBigg@{14.5}}
% \usepackage{helvet}
% \renewcommand{\familydefault}{\sfdefault}

\usetikzlibrary{shapes}
\usetikzlibrary{fit}
\usetikzlibrary{chains}
\usetikzlibrary{arrows}

% Latent node
\tikzstyle{latent} = [circle,fill=white,draw=black,inner sep=1pt,
minimum size=20pt, font=\fontsize{10}{10}\selectfont, node distance=1]
% Observed node
\tikzstyle{obs} = [latent,fill=gray!25]
% Invisible node
\tikzstyle{invisible} = [latent,minimum size=0pt,color=white, opacity=0, node distance=0]
% Constant node
\tikzstyle{const} = [rectangle, inner sep=0pt, node distance=0.1]
%state
\tikzstyle{estado} = [latent,minimum size=8pt,node distance=0.4]
%action
\tikzstyle{accion} =[latent,circle,minimum size=5pt,fill=black,node distance=0.4]


% Factor node
\tikzstyle{factor} = [rectangle, fill=black,minimum size=10pt, draw=black, inner
sep=0pt, node distance=1]
% Deterministic node
\tikzstyle{det} = [latent, rectangle]

% Plate node
\tikzstyle{plate} = [draw, rectangle, rounded corners, fit=#1]
% Invisible wrapper node
\tikzstyle{wrap} = [inner sep=0pt, fit=#1]
% Gate
\tikzstyle{gate} = [draw, rectangle, dashed, fit=#1]

% Caption node
\tikzstyle{caption} = [font=\footnotesize, node distance=0] %
\tikzstyle{plate caption} = [caption, node distance=0, inner sep=0pt,
below left=5pt and 0pt of #1.south east] %
\tikzstyle{factor caption} = [caption] %
\tikzstyle{every label} += [caption] %

\tikzset{>={triangle 45}}

%\pgfdeclarelayer{b}
%\pgfdeclarelayer{f}
%\pgfsetlayers{b,main,f}

% \factoredge [options] {inputs} {factors} {outputs}
\newcommand{\factoredge}[4][]{ %
  % Connect all nodes #2 to all nodes #4 via all factors #3.
  \foreach \f in {#3} { %
    \foreach \x in {#2} { %
      \path (\x) edge[-,#1] (\f) ; %
      %\draw[-,#1] (\x) edge[-] (\f) ; %
    } ;
    \foreach \y in {#4} { %
      \path (\f) edge[->,#1] (\y) ; %
      %\draw[->,#1] (\f) -- (\y) ; %
    } ;
  } ;
}

% \edge [options] {inputs} {outputs}
\newcommand{\edge}[3][]{ %
  % Connect all nodes #2 to all nodes #3.
  \foreach \x in {#2} { %
    \foreach \y in {#3} { %
      \path (\x) edge [->,#1] (\y) ;%
      %\draw[->,#1] (\x) -- (\y) ;%
    } ;
  } ;
}

% \factor [options] {name} {caption} {inputs} {outputs}
\newcommand{\factor}[5][]{ %
  % Draw the factor node. Use alias to allow empty names.
  \node[factor, label={[name=#2-caption]#3}, name=#2, #1,
  alias=#2-alias] {} ; %
  % Connect all inputs to outputs via this factor
  \factoredge {#4} {#2-alias} {#5} ; %
}

% \plate [options] {name} {fitlist} {caption}
\newcommand{\plate}[4][]{ %
  \node[wrap=#3] (#2-wrap) {}; %
  \node[plate caption=#2-wrap] (#2-caption) {#4}; %
  \node[plate=(#2-wrap)(#2-caption), #1] (#2) {}; %
}

% \gate [options] {name} {fitlist} {inputs}
\newcommand{\gate}[4][]{ %
  \node[gate=#3, name=#2, #1, alias=#2-alias] {}; %
  \foreach \x in {#4} { %
    \draw [-*,thick] (\x) -- (#2-alias); %
  } ;%
}

% \vgate {name} {fitlist-left} {caption-left} {fitlist-right}
% {caption-right} {inputs}
\newcommand{\vgate}[6]{ %
  % Wrap the left and right parts
  \node[wrap=#2] (#1-left) {}; %
  \node[wrap=#4] (#1-right) {}; %
  % Draw the gate
  \node[gate=(#1-left)(#1-right)] (#1) {}; %
  % Add captions
  \node[caption, below left=of #1.north ] (#1-left-caption)
  {#3}; %
  \node[caption, below right=of #1.north ] (#1-right-caption)
  {#5}; %
  % Draw middle separation
  \draw [-, dashed] (#1.north) -- (#1.south); %
  % Draw inputs
  \foreach \x in {#6} { %
    \draw [-*,thick] (\x) -- (#1); %
  } ;%
}

% \hgate {name} {fitlist-top} {caption-top} {fitlist-bottom}
% {caption-bottom} {inputs}
\newcommand{\hgate}[6]{ %
  % Wrap the left and right parts
  \node[wrap=#2] (#1-top) {}; %
  \node[wrap=#4] (#1-bottom) {}; %
  % Draw the gate
  \node[gate=(#1-top)(#1-bottom)] (#1) {}; %
  % Add captions
  \node[caption, above right=of #1.west ] (#1-top-caption)
  {#3}; %
  \node[caption, below right=of #1.west ] (#1-bottom-caption)
  {#5}; %
  % Draw middle separation
  \draw [-, dashed] (#1.west) -- (#1.east); %
  % Draw inputs
  \foreach \x in {#6} { %
    \draw [-*,thick] (\x) -- (#1); %
  } ;%
}


 \mode<presentation>
 {
 %   \usetheme{Madrid}      % or try Darmstadt, Madrid, Warsaw, ...
 %   \usecolortheme{default} % or try albatross, beaver, crane, ...
 %   \usefonttheme{serif}  % or try serif, structurebold, ...
  \usetheme{Antibes}
  \setbeamertemplate{navigation symbols}{}
 }
\estrue
\usepackage{todonotes}
\setbeameroption{show notes}
%
\newcommand{\gray}{\color{black!55}}
\usepackage{ulem} % sout
\usepackage{mdframed}
\usepackage{comment}
\usepackage{listings}
\lstset{
  aboveskip=3mm,
  belowskip=3mm,
  showstringspaces=true,
  columns=flexible,
  basicstyle={\ttfamily},
  breaklines=true,
  breakatwhitespace=true,
  tabsize=4,
  showlines=true
}


\begin{document}

\color{black!85}
\large

\begin{frame}[plain,noframenumbering]


\begin{textblock}{160}(0,0)
\includegraphics[width=1\textwidth]{static/deforestacion}
\end{textblock}

\begin{textblock}{80}(22,10)
\textcolor{black!15}{\fontsize{44}{55}\selectfont Causal}
\end{textblock}

\begin{textblock}{47}(95,70)
\centering \textcolor{black!15}{{\fontsize{52}{65}\selectfont Artificial}}
\end{textblock}

\begin{textblock}{80}(96,28)
\LARGE  \textcolor{black!15}{\rotatebox[origin=tr]{-3}{\scalebox{9}{\scalebox{1}[-1]{$p$}}}}
\end{textblock}


\begin{textblock}{80}(66,43)
\LARGE  \textcolor{black!15}{\scalebox{6}{$=$}}
\end{textblock}

\begin{textblock}{80}(36,29)
\LARGE  \textcolor{black!15}{\scalebox{9}{$p$}}
\end{textblock}

 \vspace{2cm}
\maketitle



\begin{textblock}{160}(01,67)
\normalsize \textcolor{black!5}{Inteligencia Artificial vs Causal}
\end{textblock}

% Lugar
\begin{textblock}{160}(01,73)
\scriptsize \textcolor{black!5}{
Gustavo Landfried. \\
Jueves 12 de septiembre 2024 \\
Laboratorios de Métodos Bayesianos. \\
Facultad de Ciencias Exactas y Naturales \\
Universidad de Buenos Aires, Argentina}
\end{textblock}

\end{frame}



\begin{frame}[plain]
\begin{textblock}{170}(00,0) \centering
\includegraphics[width=1\textwidth]{static/agua.jpg}
\end{textblock}


\begin{textblock}{160}(00,04) \centering
\LARGE Las cualidades de los sistemas naturales \\ \Large
emergen de la interacción entre sus partes
\end{textblock}
\vspace{1cm} \Large



\begin{textblock}{155}(00,26) \centering
\includegraphics[width=0.7\textwidth]{static/atomo_de_oxigeno.png}
\end{textblock}

\end{frame}

\begin{frame}[plain]
\begin{textblock}{160}(0,4)
 \centering \LARGE La complejidad de la vida \\ \Large
\only<5>{Células que viven en células}\only<6>{Organismos multicelulares}\only<7>{Sociedades}\only<8>{Ecosistemas}
\end{textblock}

\only<1-4>{
\begin{textblock}{160}(0,25)\centering
\scalebox{1.33}{
\tikz{
    \node[accion] (i1) {} ;
    \node[accion, yshift=0.6cm, xshift=0.4cm] (i2) {} ;
    \node[accion, yshift=0.6cm, xshift=-0.4cm] (i3) {} ;
    \node[const, yshift=0.3cm, xshift=0.4cm] (i) {};

    \node[const, yshift=-0.8cm] (ni) {$\hfrac{\text{Individuos}}{\text{solitarios}}$};

    \onslide<2->{
    \node[const, yshift=1.2cm, xshift=1.5cm] (m1) {$\hfrac{\text{Formación}}{\text{de grupos}}$};

    \node[const, right=of i, xshift=2cm] (c) {};
    \node[accion, below=of c, yshift=0.35cm, xshift=0.4cm] (c1) {} ;
    \node[accion, above=of c, yshift=-0.35cm, xshift=0.6cm] (c2) {} ;
    \node[accion, above=of c, yshift=-0.35cm, xshift=0.2cm] (c3) {} ;
    \node[const, right=of c, xshift=0.6cm] (cc) {};
    \node[const, right=of ni, xshift=1.3cm] (nc) {$\hfrac{\text{Grupos}}{\text{cooperativos}}$};
    }

    \onslide<3->{

    \node[const, right=of m1, xshift=1.2cm] (m2) {$\hfrac{\text{Transición}}{\text{mayor}}$};

    \node[const, right=of cc, xshift=2cm] (t) {};
    \node[accion, below=of t, yshift=0.35cm, xshift=0.4cm] (t1) {} ;
    \node[accion, above=of t, yshift=-0.35cm, xshift=0.6cm] (t2) {} ;
    \node[accion, above=of t, yshift=-0.35cm, xshift=0.2cm] (t3) {} ;

    \node[const, right=of nc, xshift=1.1cm] (nt) {$\hfrac{\text{Unidad de}}{\text{nivel superior}}$};
    }

    \onslide<2->{
    \edge {i} {c};
    }

    \onslide<3->{
    \edge {cc} {t};

    \plate {transition} {(t1)(t2)(t3)} {}; %
    }
}
}

\vspace{1cm}
\Large
\only<4>{Nuestra propia vida dependen de 4 niveles de cooperación.}
\end{textblock}
}

\only<5>{
\begin{textblock}{160}(0,20)\centering
\includegraphics[width=1.05\textwidth]{static/cloroplastos}
\end{textblock}
}

\only<6>{
\begin{textblock}{160}(0,20)\centering
\includegraphics[width=1\textwidth]{static/fotosintesis}
\end{textblock}
}

\only<7>{
\begin{textblock}{160}(0,20)\centering
\includegraphics[width=1\textwidth]{static/hormigas2}
\end{textblock}
}

\only<8>{
\begin{textblock}{160}(0,20)\centering
\includegraphics[width=1\textwidth]{static/tsimane}
\end{textblock}
}


\only<9-10>{
\begin{textblock}{150}(05,20)\centering
\includegraphics[width=1\textwidth]{static/biomassBarOn.png}
\end{textblock}
}

\only<9>{
\begin{textblock}{150}(62.8,20)
\scalebox{1.33}{
\tikz{
  \node[factor, color=white, minimum size=4cm] (a1) {};
}}
\end{textblock}
}


\only<9>{
\begin{textblock}{150}(80,20)
\scalebox{1.33}{
\tikz{
  \node[factor, color=white, minimum size=8cm] (a1) {};
}}
\end{textblock}
}


\end{frame}


\begin{frame}[plain]
\only<1>{
\begin{textblock}{160}(0,10)\centering
 \rotatebox{-90}{
 \includegraphics[width=0.51\textwidth]{static/cerebros.jpg}}
\end{textblock}
}
\only<2>{
\begin{textblock}{160}(0,10)\centering
 \includegraphics[width=1\textwidth]{static/cerebros_red.jpg}
\end{textblock}
}

\only<3>{
\begin{textblock}{160}(0,-16)
\includegraphics[width=1\textwidth]{static/madre-chimpance.jpg}
\end{textblock}

\begin{textblock}{80}(80,4)
 \centering \LARGE Crianza cooperativa \\
 \Large Coevolución genético-cultural
\end{textblock}

}


\only<1-2>{
\begin{textblock}{160}(0,4)
 \centering \LARGE La Inteligencia Humana \\ \large
\only<1>{Hipótesis cognitiva}\only<2>{Hipótesis cultural}
\end{textblock}
}



\end{frame}


\begin{frame}[plain,noframenumbering]
\centering \vspace{0.5cm}
\includegraphics[width=0.66\textwidth]{static/Metodos2.png}
\end{frame}











\end{document}



