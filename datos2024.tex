\newif\ifen
\newif\ifes
\newif\iffr
\newcommand{\fr}[1]{\iffr#1 \fi}
\newcommand{\En}[1]{\ifen#1\fi}
\newcommand{\Es}[1]{\ifes#1\fi}
\estrue
\documentclass[shownotes,aspectratio=169]{beamer}

\usepackage{siunitx}
\input{diapo_encabezado.tex}
\input{tikzlibrarybayesnet.code.tex}
 \mode<presentation>
 {
 %   \usetheme{Madrid}      % or try Darmstadt, Madrid, Warsaw, ...
 %   \usecolortheme{default} % or try albatross, beaver, crane, ...
 %   \usefonttheme{serif}  % or try serif, structurebold, ...
  \usetheme{Antibes}
  \setbeamertemplate{navigation symbols}{}
 }
\estrue
\usepackage{todonotes}
\setbeameroption{show notes}
%
\newcommand{\gray}{\color{black!55}}
\usepackage{ulem} % sout
\usepackage{mdframed}
\usepackage{comment}
\usepackage{listings}
\lstset{
  aboveskip=3mm,
  belowskip=3mm,
  showstringspaces=true,
  columns=flexible,
  basicstyle={\ttfamily},
  breaklines=true,
  breakatwhitespace=true,
  tabsize=4,
  showlines=true
}


\begin{document}

\color{black!85}
\large

\begin{frame}[plain,noframenumbering]


\begin{textblock}{160}(0,0)
\includegraphics[width=1\textwidth]{static/deforestacion}
\end{textblock}

\begin{textblock}{80}(22,10)
\textcolor{black!15}{\fontsize{44}{55}\selectfont Causal}
\end{textblock}

\begin{textblock}{47}(95,70)
\centering \textcolor{black!15}{{\fontsize{52}{65}\selectfont Artificial}}
\end{textblock}

\begin{textblock}{80}(96,28)
\LARGE  \textcolor{black!15}{\rotatebox[origin=tr]{-3}{\scalebox{9}{\scalebox{1}[-1]{$p$}}}}
\end{textblock}


\begin{textblock}{80}(66,43)
\LARGE  \textcolor{black!15}{\scalebox{6}{$=$}}
\end{textblock}

\begin{textblock}{80}(36,29)
\LARGE  \textcolor{black!15}{\scalebox{9}{$p$}}
\end{textblock}

 \vspace{2cm}
\maketitle



\begin{textblock}{160}(01,67)
\normalsize \textcolor{black!5}{Inteligencia Artificial vs Causal}
\end{textblock}

% Lugar
\begin{textblock}{160}(01,73)
\scriptsize \textcolor{black!5}{
Gustavo Landfried. \\
Jueves 12 de septiembre 2024 \\
Laboratorios de Métodos Bayesianos. \\
Facultad de Ciencias Exactas y Naturales \\
Universidad de Buenos Aires, Argentina}
\end{textblock}

\end{frame}



\begin{frame}[plain]
\begin{textblock}{170}(00,0) \centering
\includegraphics[width=1\textwidth]{static/agua.jpg}
\end{textblock}


\begin{textblock}{160}(00,04) \centering
\LARGE Las cualidades de los sistemas naturales \\ \Large
emergen de la interacción entre sus partes
\end{textblock}
\vspace{1cm} \Large



\begin{textblock}{155}(00,26) \centering
\includegraphics[width=0.7\textwidth]{static/atomo_de_oxigeno.png}
\end{textblock}

\end{frame}

\begin{frame}[plain]
\begin{textblock}{160}(0,4)
 \centering \LARGE La complejidad de la vida \\ \Large
\only<5>{Células que viven en células}\only<6>{Organismos multicelulares}\only<7>{Sociedades}\only<8>{Ecosistemas}
\end{textblock}

\only<1-4>{
\begin{textblock}{160}(0,25)\centering
\scalebox{1.33}{
\tikz{
    \node[accion] (i1) {} ;
    \node[accion, yshift=0.6cm, xshift=0.4cm] (i2) {} ;
    \node[accion, yshift=0.6cm, xshift=-0.4cm] (i3) {} ;
    \node[const, yshift=0.3cm, xshift=0.4cm] (i) {};

    \node[const, yshift=-0.8cm] (ni) {$\hfrac{\text{Individuos}}{\text{solitarios}}$};

    \onslide<2->{
    \node[const, yshift=1.2cm, xshift=1.5cm] (m1) {$\hfrac{\text{Formación}}{\text{de grupos}}$};

    \node[const, right=of i, xshift=2cm] (c) {};
    \node[accion, below=of c, yshift=0.35cm, xshift=0.4cm] (c1) {} ;
    \node[accion, above=of c, yshift=-0.35cm, xshift=0.6cm] (c2) {} ;
    \node[accion, above=of c, yshift=-0.35cm, xshift=0.2cm] (c3) {} ;
    \node[const, right=of c, xshift=0.6cm] (cc) {};
    \node[const, right=of ni, xshift=1.3cm] (nc) {$\hfrac{\text{Grupos}}{\text{cooperativos}}$};
    }

    \onslide<3->{

    \node[const, right=of m1, xshift=1.2cm] (m2) {$\hfrac{\text{Transición}}{\text{mayor}}$};

    \node[const, right=of cc, xshift=2cm] (t) {};
    \node[accion, below=of t, yshift=0.35cm, xshift=0.4cm] (t1) {} ;
    \node[accion, above=of t, yshift=-0.35cm, xshift=0.6cm] (t2) {} ;
    \node[accion, above=of t, yshift=-0.35cm, xshift=0.2cm] (t3) {} ;

    \node[const, right=of nc, xshift=1.1cm] (nt) {$\hfrac{\text{Unidad de}}{\text{nivel superior}}$};
    }

    \onslide<2->{
    \edge {i} {c};
    }

    \onslide<3->{
    \edge {cc} {t};

    \plate {transition} {(t1)(t2)(t3)} {}; %
    }
}
}

\vspace{1cm}
\Large
\only<4>{Nuestra propia vida dependen de 4 niveles de cooperación.}
\end{textblock}
}

\only<5>{
\begin{textblock}{160}(0,20)\centering
\includegraphics[width=1.05\textwidth]{static/cloroplastos}
\end{textblock}
}

\only<6>{
\begin{textblock}{160}(0,20)\centering
\includegraphics[width=1\textwidth]{static/fotosintesis}
\end{textblock}
}

\only<7>{
\begin{textblock}{160}(0,20)\centering
\includegraphics[width=1\textwidth]{static/hormigas2}
\end{textblock}
}

\only<8>{
\begin{textblock}{160}(0,20)\centering
\includegraphics[width=1\textwidth]{static/tsimane}
\end{textblock}
}


\only<9-10>{
\begin{textblock}{150}(05,20)\centering
\includegraphics[width=1\textwidth]{static/biomassBarOn.png}
\end{textblock}
}

\only<9>{
\begin{textblock}{150}(62.8,20)
\scalebox{1.33}{
\tikz{
  \node[factor, color=white, minimum size=4cm] (a1) {};
}}
\end{textblock}
}


\only<9>{
\begin{textblock}{150}(80,20)
\scalebox{1.33}{
\tikz{
  \node[factor, color=white, minimum size=8cm] (a1) {};
}}
\end{textblock}
}


\end{frame}


\begin{frame}[plain]
\only<1>{
\begin{textblock}{160}(0,10)\centering
 \rotatebox{-90}{
 \includegraphics[width=0.51\textwidth]{static/cerebros.jpg}}
\end{textblock}
}
\only<2>{
\begin{textblock}{160}(0,10)\centering
 \includegraphics[width=1\textwidth]{static/cerebros_red.jpg}
\end{textblock}
}

\only<3>{
\begin{textblock}{160}(0,-16)
\includegraphics[width=1\textwidth]{static/madre-chimpance.jpg}
\end{textblock}

\begin{textblock}{80}(80,4)
 \centering \LARGE Crianza cooperativa \\
 \Large Coevolución genético-cultural
\end{textblock}

}

\only<1-2>{
\begin{textblock}{160}(0,4)
 \centering \LARGE La Inteligencia Humana \\ \large
\only<1>{Hipótesis cognitiva}\only<2>{Hipótesis cultural}
\end{textblock}
}


\only<4-6>{
 \begin{textblock}{150}(2,16)
 \includegraphics[width=1\textwidth]{static/mapa_icosaedro_blanco.png}
 \end{textblock}

\only<5->{
 \begin{textblock}{160}(100,43)
  \tikz{%Norte-Sur
    \node[const] (norte) {};
    \node[const, xshift=0.8cm,yshift=0.75cm] (sur) {};
    \path[draw, line width=0.08cm, -latex, fill=black!50] (norte) edge[bend left,draw=black!50] node[right,color=black!75] {} (sur);
  }
 \end{textblock}
}

  \only<4->{
 \begin{textblock}{160}(50,25)
  \tikz{% Africa-Mundo
    \node[const] (africa) {};
    \node[const, xshift=0.55cm,yshift=-1.4cm] (asia) {};
    \path[draw, line width=0.08cm, -latex, fill=black!50] (africa) edge[bend left,draw=black!50] node[right,color=black!75] {} (asia);
  }
 \end{textblock}
 }

  \only<5->{
 \begin{textblock}{160}(60,59)
  \tikz{%China-America
    \node[const] (asia) {};
    \node[const, xshift=1.55cm,yshift=0cm] (america) {};
    \path[draw, line width=0.08cm, -latex, fill=black!50] (asia) edge[bend right,draw=black!50] node[right,color=black!75] {} (america);
  }
 \end{textblock}
  }

  \only<5->{
 \begin{textblock}{160}(40,59)
  \tikz{%China-Oceania
    \node[const] (asia) {};
    \node[const, xshift=-1.33cm,yshift=-1cm] (oceania) {};
    \path[draw, line width=0.08cm, -latex, fill=black!50] (asia) edge[bend right,draw=black!50] node[right,color=black!75] {} (oceania);
  }
 \end{textblock}
  }


%
%  \begin{textblock}{128}(41,40)
%   \tikz{%India
%     \node[latent, fill=red,minimum size=0.15cm] () {};
%   }
%  \end{textblock}

\only<6->{
  \begin{textblock}{160}(112,42)
  \tikz{%Peru
    \node[latent, fill=red,minimum size=0.15cm] () {};
  }
 \end{textblock}
 \begin{textblock}{160}(98,52.3)
  \tikz{%Mexico
    \node[latent, fill=red,minimum size=0.15cm] () {};
  }
 \end{textblock}
 \begin{textblock}{160}(48,24)
  \tikz{%Subshara
    \node[latent, fill=red,minimum size=0.15cm] () {};
  }
 \end{textblock}
 \begin{textblock}{160}(54,38)
  \tikz{%Irak
    \node[latent, fill=red,minimum size=0.15cm] () {};
  }
 \end{textblock}
 \begin{textblock}{150}(56,59)
  \tikz{%China
    \node[latent, fill=red,minimum size=0.15cm] () {};
  }
 \end{textblock}
 \begin{textblock}{150}(39,72)
  \tikz{% Nueva Guinea
    \node[latent, fill=red,minimum size=0.15cm] () {};
  }
 \end{textblock}
}

\begin{textblock}{160}(0,4)
 \centering \LARGE La Inteligencia Humana\\ \large
   \only<1->{Emergió de la \textbf{interacción} cultural}
\end{textblock}

}

% \only<7>{
% \begin{textblock}{191}(-16,0)
%  \centering
%  \includegraphics[width=1\textwidth]{static/terrazas_arroz_c.jpg}
% \end{textblock}
%
%  \begin{textblock}{120}(10,1)
%   \LARGE \textcolor{black!5}{Reciprocidad ecológica}\\
%  \end{textblock}
%
%  \begin{textblock}{70}(88,60)
%   \Large \textcolor{black!5}{Domesticación}\\
%  \end{textblock}
% }
%
% \only<8>{
% \begin{textblock}{95}(62,22) \centering
% \includegraphics[width=0.95\textwidth]{static/polynesia.png}
% \end{textblock}
%
% \begin{textblock}{60}(04,08.2) \centering
% \includegraphics[width=0.95\textwidth]{static/tonga_barco.jpg}
% \end{textblock}
%
% \begin{textblock}{160}(0,4)
% \centering \LARGE \textcolor{black!85}{Agricultura $\mapsto$ Aumento poblacional $\mapsto$ Centros de innovación }
% \end{textblock}
% }
%
%
% \only<9>{
% \begin{textblock}{140}(10,22) \centering
% \includegraphics[width=0.7\textwidth]{static/tasmania.jpg}
% \end{textblock}
% }

\end{frame}

\begin{frame}[plain]

% \only<1>{
% \begin{textblock}{160}(0,0)
% \includegraphics[width=1\textwidth]{static/pizarron.jpg}
% \end{textblock}
% }

\begin{textblock}{160}(0,4) \centering \LARGE
\only<17->{\hspace{0.19cm}} La Inteligencia \only<-16>{Humana}\only<17->{\textbf{Artificial}} \\ \large Probabilidad
\end{textblock}


 \begin{textblock}{160}(0,68)
 \centering \LARGE \only<3>{Certeza absoluta}\only<4>{Distribución de creencias \\ }\only<5>{Distribución de creencias \\ \Large Honesta }
 \only<9>{¿Cómo preservamos los acuerdos intersubjetivos?\\}
\end{textblock}
\vspace{1.5cm}
\centering

\only<1>{
\begin{textblock}{160}(0,32) \Large
\begin{align*}
P(\text{Hipótesis}|\text{Datos})
\end{align*}

\vspace{1.3cm}

\textbf{Sistema de razonamiento en contextos de incertidumbre}

\end{textblock}
}


\only<2>{
\begin{textblock}{160}(0,62)
\Large Detrás de una de estas caja hay un regalo. \\[0.1cm]

\large ¿Dónde está el regalo?
\end{textblock}
}

\only<2>{
\begin{textblock}{160}(0,28)
 \scalebox{1.1}{
\tikz{ %
         \node[factor, minimum size=1cm] (p1) {} ;
         \node[factor, minimum size=1cm, xshift=1.5cm] (p2) {} ;
         \node[factor, minimum size=1cm, xshift=3cm] (p3) {} ;


         \node[const, above=of p1, yshift=0.1cm] (np1) {\Large $?$};
         \node[const, above=of p2, yshift=0.1cm] (np2) {\Large $?$};
         \node[const, above=of p3, yshift=0.1cm] (np3) {\Large $?$};
         }
}
\end{textblock}
}

\only<3>{
\begin{textblock}{160}(0,28)
 \scalebox{1.1}{
\tikz{ %
         \node[factor, minimum size=1cm] (p1) {} ;
         \node[factor, minimum size=1cm, xshift=1.5cm] (p2) {} ;
         \node[factor, minimum size=1cm, xshift=3cm] (p3) {} ;


         \node[const, above=of p1, yshift=0.125cm] (np1) {\Large $0$};
         \node[const, above=of p2, yshift=0.125cm] (np2) {\Large $1$};
         \node[const, above=of p3, yshift=0.125cm] (np3) {\Large $0$};
         }
}
\end{textblock}
}

\only<4>{
\begin{textblock}{160}(0,28)
 \scalebox{1.1}{
\tikz{ %
         \node[factor, minimum size=1cm] (p1) {} ;
         \node[factor, minimum size=1cm, xshift=1.5cm] (p2) {} ;
         \node[factor, minimum size=1cm, xshift=3cm] (p3) {} ;


         \node[const, above=of p1, yshift=-0.05cm] (np1) {\Large $1/10$};
         \node[const, above=of p2, yshift=-0.05cm] (np2) {\Large $8/10$};
         \node[const, above=of p3, yshift=-0.05cm] (np3) {\Large $1/10$};
         }
}
\end{textblock}
}


\only<5-8>{
\begin{textblock}{160}(0,28)
 \scalebox{1.1}{
\tikz{ %
         \only<5-7>{\node[factor, minimum size=1cm] (p1) {} ;}
         \only<8>{
         \node[factor, minimum size=1cm] (p1) {\includegraphics[width=0.03\textwidth]{static/cerradura.png}} ;}
         \node[factor, minimum size=1cm, xshift=1.5cm] (p2) {} ;
         \node[factor, minimum size=1cm, xshift=3cm] (p3) {} ;


         \node[const, above=of p1, yshift=-0.05cm] (np1) {\Large $1/3$};
         \node[const, above=of p2, yshift=-0.05cm] (np2) {\Large $1/3$};
         \node[const, above=of p3, yshift=-0.05cm] (np3) {\Large $1/3$};
         }
}
\end{textblock}
}

\only<6-8>{
\begin{textblock}{150}(10,64)   \centering \Large
No mentir\\[0.1cm]
\large 1. \textbf{No asegurar más de lo que sabés} (maximizar incertidumbre) \\
\only<7->{\large 2. \textbf{Sin ocultar lo que sí sabés} (dada la información disponible)}
\end{textblock}
}

\only<9>{
\begin{textblock}{160}(0,28)
 \scalebox{1.1}{
\tikz{ %
         \node[factor, minimum size=1cm] (p1) {\includegraphics[width=0.03\textwidth]{static/cerradura.png}} ;
         \node[det, minimum size=1cm, xshift=1.5cm] (p2) {\includegraphics[width=0.03\textwidth]{static/dedo.png}} ;
         \node[factor, minimum size=1cm, xshift=3cm] (p3) {} ;


         \node[const, above=of p1, yshift=-0.05cm] (np1) {\Large $\phantom{/}?\phantom{/}$};
         \node[const, above=of p2, yshift=-0.05cm] (np2) {\Large $\phantom{/}0\phantom{/}$};
         \node[const, above=of p3, yshift=-0.05cm] (np3) {\Large $\phantom{/}?\phantom{/}$};
         }
}
\end{textblock}
}


\only<10-11>{

\begin{textblock}{80}(0,19)
\centering
\tikz{
    \node[invisible] (id) {};
    \node[latent, left=of id, xshift=0cm] (d) {\includegraphics[width=0.10\textwidth]{static/dedo.png}} ;
    %\node[const,left=of d] (nd) {\Large $P(s|r)$} ;
    \node[const, below=of d, yshift=0cm] (restricciones) {$s \neq r$};


    \node[latent, above=of d, xshift=-1.5cm] (r) {\includegraphics[width=0.12\textwidth]{static/regalo.png}} ;
    %\node[const,left=of r] (nr) {\Large $P(r)$} ;


    \node[latent, fill=black!30, above=of d, xshift=1.5cm] (c) {\includegraphics[width=0.12\textwidth]{static/cerradura.png}} ;
    %\node[const,left=of c] (nc) {\Large $P(c)$} ;

    \edge {r} {d};

    %\node[invisible, right=of c, xshift=0.6cm, yshift=0.65cm] (ic) {};

}

\vspace{0.5cm}
\tikz{
         \node[factor, minimum size=1cm] (p1) {\includegraphics[width=0.07\textwidth]{static/cerradura.png}} ;
         \node[det, minimum size=1cm, xshift=1.5cm] (p2) {\includegraphics[width=0.07\textwidth]{static/dedo.png}} ;
         \node[factor, minimum size=1cm, xshift=3cm] (p3) {} ;

         \node[const, above=of p1, yshift=.1cm] (fp1) {$1/2$};
         \node[const, above=of p2, yshift=.1cm] (fp2) {$\phantom{/}0\phantom{/}$};
         \node[const, above=of p3, yshift=.1cm] (fp3) {$1/2$};
         \node[const, below=of p2, yshift=-.10cm, xshift=0.3cm] (dedo) {};
}
\end{textblock}
}

\only<11>{
\begin{textblock}{80}(80,19)
\centering
\tikz{
    \node[invisible] (id) {};
    \node[latent, left=of id, xshift=0cm] (d) {\includegraphics[width=0.10\textwidth]{static/dedo.png}} ;
    %\node[const,left=of d] (nd) {\Large $P(s_{_{\only<-2>{\phantom}{t}}}|r_{_{\only<-2>{\phantom}{t}}},c_{_{\only<-2>{\phantom}{t}}})$} ;
    \node[const, below=of d, yshift=-0cm] (restricciones) {$s \neq r \text{, } s \neq c$};


    \node[latent, above=of d, xshift=-1.5cm] (r) {\includegraphics[width=0.12\textwidth]{static/regalo.png}} ;
    %\node[const,left=of r] (nr) {\Large $P(r_{_{\only<-2>{\phantom}{t}}})$} ;


    \node[latent, fill=black!30, above=of d, xshift=1.5cm] (c) {\includegraphics[width=0.12\textwidth]{static/cerradura.png}} ;
    %\node[const,left=of c] (nc) {\Large $P(c_{_{\only<-2>{\phantom}{t}}})$} ;

    %\node[invisible, right=of c, xshift=0.6cm, yshift=0.65cm] (ic) {};

    \edge {r,c} {d};
    %\onslide<3->{\plate {episodio} {(d)(ic)(nr)} {$t$ : Episodio};}
}

\vspace{0.5cm}
 \tikz{
         \node[factor, minimum size=1cm] (p1) {\includegraphics[width=0.07\textwidth]{static/cerradura.png}} ;
         \node[det, minimum size=1cm, xshift=1.5cm] (p2) {\includegraphics[width=0.07\textwidth]{static/dedo.png}} ;
         \node[factor, minimum size=1cm, xshift=3cm] (p3) {} ;

         \node[const, above=of p1, yshift=.1cm] (fp1) {$1/3$};
         \node[const, above=of p2, yshift=.1cm] (fp2) {$\phantom{/}0\phantom{/}$};
         \node[const, above=of p3, yshift=.1cm] (fp3) {$2/3$};
         \node[const, below=of p2, yshift=-.10cm, xshift=0.3cm] (dedo) {};
}
\end{textblock}
}



\only<12-18>{
\begin{textblock}{160}(0,23)  \centering
\begin{align*}
\onslide<14->{\underbrace{P\left(\hfrac{\text{Modelo}}{\text{causal}}, \ \hfrac{\text{Datos}}{\{d_1, d_2, \dots \}} \right)}_{\hfrac{\textbf{\small Creencia compatible}}{\textbf{\small con los datos} }} =} \onslide<12->{\underbrace{P\left(\hfrac{\text{Modelo}}{\text{causal}}\right)}_{\hfrac{\text{\small \phantom{p}Creencia inicial\phantom{p}}}{\text{\small}}}} \onslide<13->{ \underbrace{P\left(d_1\bigg|\hfrac{\text{Modelo}}{\text{causal}}\right)P\left(d_2\bigg|d_1,\hfrac{\text{Modelo}}{\text{causal}}\right) \dots}_{\hfrac{\text{\small \only<15->{\textbf}{\phantom{p}Secuencia de predicciones\phantom{p}}}}{\text{\small \only<15->{\textbf}{realizada por el modelo causal}} }}}
\end{align*}
\end{textblock}
}


\only<15>{
\begin{textblock}{160}(0,60)  \centering \Large
Un cero en la secuencia de predicciones \\

hace falso al modelo causal para siempre.
\end{textblock}
}
%
\only<16>{
\begin{textblock}{120}(30,56)  \centering
\begin{flalign*}
 &P\left(d_1\bigg|\hfrac{\text{Modelo}}{\text{causal}}\right) =  \onslide<16>{\underbrace{\underset{\hfrac{\text{Hipótesis}}{\text{(en modelo causal)}}}{\text{Suma}}(\ P(d_1|\text{Hipótesis})P(\text{Hipótesis}) \ )}_{\hfrac{\text{\small \phantom{p}Predicción con la contribución\phantom{p}}}{\text{\small de todas sus hipótesis internas}} }} &&
\end{flalign*}
%\hfrac{\text{\footnotesize \phantom{p}Predicción de sus\phantom{p}} }{\text{\footnotesize hipótesis internas}}}
\end{textblock}
}

\only<17-18>{
\begin{textblock}{120}(30,56)  \centering
\begin{flalign*}
 & P\left(d_1\bigg|\hfrac{\text{Modelo}}{\text{causal}}\right) = \hspace{2.23cm} P(d_1|\text{Hipótesis}) &&
\end{flalign*}
\end{textblock}
}
\only<18>{
\begin{textblock}{160}(0,78)  \centering \Large
\textbf{Selecciona arbitrariamente una única hipótesis}
\end{textblock}
}


\only<19>{
\begin{textblock}{160}(0,36)  \centering \Large
Si nos va mejor en el entrenamiento \\ nos debería ir mejor en la evaluación
\end{textblock}
}


\only<20>{
\begin{textblock}{160}(0,22)  \centering
\textbf{Efectos secundarios indeseados}

\includegraphics[width=0.7\textwidth]{code/overfitting.pdf}
\end{textblock}
}


\end{frame}

\begin{frame}[plain]
\begin{textblock}{160}(0,4)  \centering \LARGE
La Inteligencia de la Vida\\ \large
 \only<1-3>{Evolución}\only<4->{Apuestas de vida}
\end{textblock}

\only<1-3>{
\begin{equation*}
\underbrace{\text{P}(\text{Variante},\text{\En{Data}\Es{Datos}})}_{\hfrac{\text{\footnotesize\En{Initial belief compatible}\Es{Tamaño actual}}}{\text{\footnotesize \En{with the data}\Es{de la población}}}} = \underbrace{\text{P}(\text{Variante})}_{\hfrac{\text{\footnotesize\En{Initial intersubjective}\Es{Tamaño inicial}}}{\text{\footnotesize\En{agreement}\Es{de la población}}}} \underbrace{\text{ R}(\text{dato}_1|\text{Variante})}_{\text{\footnotesize Reproducción $\geq 1$}} \, \underbrace{\text{ S}(\text{dato}_2|\text{dato}_1,\text{Variante}) }_{\text{\footnotesize $0 \leq$ Supervivencia $\leq 1$  }} \dots
\end{equation*}
}

\only<2-3>{
\begin{textblock}{160}(0,60) \centering \Large
Un 0 en la secuencia de tasas de reproducción y

supervivencia produce una extinción irreversible

\vspace{0.6cm}

\only<3>{
\textbf{¿Cuáles son las variantes que más crecen?}
}
\end{textblock}
}



\only<4-13>{
\begin{textblock}{150}(05,32)
Casa de apuestas paga: \\[0.2cm] \normalsize

\ \ $\bullet$ Por \textbf{Cara}. $Q_c = 3$

\ \ $\bullet$ Por \textbf{Sello}. $Q_s = 1.2$

\vspace{0.3cm} \large

Invertimos nuestra energía\\[0.2cm] \normalsize

\ \ $\bullet$ A \textbf{Cara}. $b\only<4-5>{\in [0,1]}\only<6->{=0.5}$

\ \ $\bullet$ A \textbf{Sello}. $(1-b)$


\end{textblock}
}


\only<4>{
\begin{textblock}{95}(60,32)
\raggedleft
\includegraphics[width=0.82\textwidth]{static/plata-potosi.jpg}
\end{textblock}
}

\only<5-6>{
\begin{textblock}{95}(60,36) \centering \Large
Si invertimos todo a una opción y sale \\

la otra, perdemos todos los recursos \\

\only<6>{
\vspace{1cm}

¡Tenemos que diversificar!

($b=0.5$)

}
\end{textblock}
}

\only<7>{
\begin{textblock}{105}(50,36) \centering \Large
\begin{flalign*}
& \text{Si sale Cara: } 50 \cdot Q_c = 150 \text{ \ \ \ 50\%} \\
& \text{Si sale Sello: }50 \cdot Q_s = 60 \text{ \ \ \  -40\%}
\end{flalign*}
\end{textblock}
}

\only<8>{
\begin{textblock}{95}(60,24) \centering \Large
\includegraphics[page=6,width=1\textwidth]{code/apuestasParalelas.pdf}
\end{textblock}
}
\only<9>{
\begin{textblock}{95}(60,24) \centering \Large
\includegraphics[page=7,width=1\textwidth]{code/apuestasParalelas.pdf}
\end{textblock}
}
\only<10>{
\begin{textblock}{95}(60,24) \centering \Large
\includegraphics[page=8,width=1\textwidth]{code/apuestasParalelas.pdf}
\end{textblock}
}
\only<11>{
\begin{textblock}{95}(60,24) \centering \Large
\includegraphics[page=10,width=1\textwidth]{code/apuestasParalelas.pdf}
\end{textblock}
}
\only<12>{
\begin{textblock}{95}(60,24) \centering \Large
\includegraphics[page=11,width=1\textwidth]{code/apuestasParalelas.pdf}
\end{textblock}
}
\only<13->{
\begin{textblock}{95}(60,24) \centering \Large
\includegraphics[page=13,width=1\textwidth]{code/apuestasParalelas.pdf}
\end{textblock}
}

\only<14->{
\begin{textblock}{60}(0,42) \centering \Large
Los impactos de las caídas \\

son difíciles de recuperar \\[0.1cm] \large

(de la extinción no se vuelve)
\end{textblock}
}

\end{frame}



\begin{frame}[plain]
\begin{textblock}{160}(0,4)
\centering \LARGE La estrategia de la vida \\
\large La cooperación
\end{textblock}



\only<1-6>{
\begin{textblock}{160}(0,22) \centering \Large


La emergencia de unidades cooperativas de

nivel superior es un fenómeno permanente

\vspace{0.8cm} \large

\only<2-6>{
Nuestra propia vida depende de al menos 4 niveles:
\begin{figure}[H]
\centering
 \begin{subfigure}[b]{0.25\textwidth} \centering
 \onslide<3->{\includegraphics[width=1\linewidth]{static/cloroplastos.jpg}
  \caption*{\En{Eukaryotic cells}\Es{Células eucariota}}}
  \end{subfigure}
 \begin{subfigure}[b]{0.23\textwidth} \centering
  \onslide<4->{\includegraphics[width=1\linewidth]{static/fotosintesis.jpg}
  \caption*{\En{Organisms}\Es{Organismos}}}
  \end{subfigure}
  \begin{subfigure}[b]{0.235\textwidth} \centering
 \onslide<5->{\includegraphics[width=1\linewidth]{static/hormigas2.jpg}
  \caption*{\En{Societies}\Es{Sociedades}}}
 \end{subfigure}
 \begin{subfigure}[b]{0.235\textwidth} \centering
 \onslide<6->{\includegraphics[width=1\linewidth]{static/tsimane2.jpg}
  \caption*{\En{Ecosystems}\Es{Ecosistemas}}}
 \end{subfigure}
\end{figure}
}
\end{textblock}
}


\only<7>{
\begin{textblock}{160}(0,24)
\centering
  \begin{tabular}{|l|c|c|c|c|c|}
     \hline
         & {\small $t=0$} & {\small \  $\Delta$}  & {\small \, $t=1$ } & {\small \  $\Delta$}  & {\small \,  $t=2$ }  \\ \hline \hline
        A no-coop& $1$ &  \ 50\% &  $1.5$ &  -40\%  & $\bm{0.9}$ \\ \hline
        B no-coop & $1$ & -40\% & $0.6$ & \ 50\%  & $\bm{0.9}$ \\ \hline
\end{tabular}
\end{textblock}
}

\only<8>{
\begin{textblock}{160}(0,24)
\centering
  \begin{tabular}{|l|c|c|c|c|c|}
     \hline
         & {\small $t=0$} & {\small \  $\Delta$}  & {\small \, $t=1$ } & {\small \  $\Delta$}  & {\small \,  $t=2$ }  \\ \hline \hline
        A no-coop& $1$ & \ 50\% &  $1.5$ & -40\% & $\bm{0.9}$ \\ \hline
        B no-coop & $1$ &  -40\% & $0.6$ & \ 50\% & $\bm{0.9}$ \\ \hline\hline
        A coop & $1$ & \ 50\% & $1.05$ & -40\% & $\bm{1.1}$ \\ \hline
        B coop & $1$ & -40\% & $1.05$ & \ 50\% & $\bm{1.1}$\\ \hline
\end{tabular}
\end{textblock}
}


\only<9>{
\begin{textblock}{100}(30,20) \centering
\includegraphics[page=10,width=1\textwidth]{figuras/apuestasParalelas.pdf}
\end{textblock}
}

\only<9>{
\begin{textblock}{100}(130,22)
Cooperación
\end{textblock}
}
\only<9>{
\begin{textblock}{100}(130,68)
Individual
\end{textblock}
}



\end{frame}



%
% \begin{frame}[plain]
% \only<1>{
% \begin{textblock}{160}(-05,0)  \centering
% \includegraphics[width=1.1\textwidth]{static/chatGPT.jpeg}
% \end{textblock}
% }
% \begin{textblock}{160}(0,4)  \centering \LARGE
% \only<1>{\textcolor{white}{Inteligencia Artificial}}%
% \only<2>{Inteligencia Artificial}
% \end{textblock}
%
%

% \end{frame}

\begin{frame}[plain,noframenumbering]
\centering \vspace{0.5cm}
\includegraphics[width=0.66\textwidth]{static/Metodos2.png}
\end{frame}











\end{document}



